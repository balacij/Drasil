\chapter{Theories}
\label{chap:modelkinds}

\begin{writingdirectives}
    \item What are theories used for in Drasil?
    \item How are they captured in Drasil?
    \item Current problems? Solution?
\end{writingdirectives}

In this chapter, we will focus on improving inspection and interpretation
capabilities of theories in Drasil. Specifically, with focus on interpreting
them for generating software artifacts.

\section{Transforming Theories to Code}
\label{chap:modelkinds:sec:transforming-theories-to-code}

As already mentioned in \Cref{chap:drasil}, the \acs{srs} template
\cite{SmithAndLai2005} breaks up software requirements and problems into a
series of well-understood components, providing developers with concrete
solution requirements they must satisfy, and domain experts with justification
for problem solutions. Notably, the \acs{srs} relates a programs \textit{inputs}
to a set of \textit{outputs} using a set of \textit{theories}. The inputs and
outputs are sets of variables, with data that need to be somehow fed into the
program, or calculated and output by the program. The theories connect the input
variables to the output by forming a \textit{solution/calculation path}.
Notably, the \ACF{im} \cite{SmithAndLai2005} theories together largely form the
calculation path, while other background theories provide justification for
them.

\subsection{An Example}
\label{chap:modelkinds:sec:transforming-theories-to-code:subsec:an-example}

For example, Drasil's \porthref{\acs{projectile} case
    study}{https://jacquescarette.github.io/Drasil/examples/projectile/SRS/srs/Projectile_SRS.html}
describes how to estimate if a launcher, aligned at a particular angle, will hit
a target from a specific distance. The \acs{srs} requires users to fill in the:
\begin{enumerate}
    \item input variables:
          \begin{enumerate}
              \item \(p_\mathit{target}\), the targets distance from the
                    launcher,
              \item \(v_\mathit{launch}\), the projectile launch speed,
              \item and \(\theta\), the launch angle.
          \end{enumerate}
    \item output variables:
          \begin{enumerate}
              \item \(s\), a message, explaining if the projectile hit the
                    target, fell short, or went long,
              \item and \(d_\mathit{offset}\), the expected distance between the
                    target position and the landing position.
          \end{enumerate}
    \item and theories, connecting the inputs to the outputs:
          \begin{enumerate}
              \item \({t_{\text{flight}}}=\frac{2 {v_{\text{launch}}}
                        \sin\left(\theta{}\right)}{\mathbf{g}}\), estimating
                    flight time with \(v_{\mathit{launch}}\) and \(\theta\),
              \item \({p_{\text{land}}}=\frac{2 {v_{\text{launch}}}^{2}
                        \sin\left(\theta{}\right)
                        \cos\left(\theta{}\right)}{\mathbf{g}}\), a calculation
                    of the landing position,
              \item \({d_{\text{offset}}}={p_{\text{land}}}-{p_{\text{target}}}\),
                    calculation of distance between the targets position and the
                    expected landing position of the projectile,
              \item and \(s=\begin{cases} \text{``The target was hit.''},        &
              |\frac{{d_{\text{offset}}}}{{p_{\text{target}}}}| < \varepsilon{}
              \\
              \text{``The projectile fell short.''}, & {d_{\text{offset}}} < 0 \\
              \text{``The projectile went long.''},  & {d_{\text{offset}}} >
              0\end{cases}\), \newline{}calculating the output message.
          \end{enumerate}
\end{enumerate}

From these 3 key bodies of information along with some supporting background
knowledge (such as assumptions, constants, etc.), Drasil forms a calculation
path, deriving the output variables from the input variables using the concrete
theories\footnote{The ``concrete theories'' are the ``instanced model'' and
    ``data definitions'' \cite{SmithAndLai2005} encodings (respectively,
    \InstanceModel{}s and \DataDefinition{}s) which are somehow usable in the
    solution of a scientific problem.}. Together, the theories form a calculation
algorithm which Drasil generates representational code of. For example,
\acsp{projectile} generates \refOriginalJavaProjectileMain{} for one of the
Java-flavoured artifacts. In \refOriginalJavaProjectileMain{}, it uses a
\inlineCode{java}{write_output} method to output the calculated output variables
after calculating them using the relevant theories\footnote{Note:
    \(t_\text{flight}\) is seemingly unused in the generated code, but it is used in
    the derivation of \(p_\text{land}\). However, it being ``unused'' is irrelevant
    to this work.}.

\originalJavaProjectileMain{}

As seen in \refOriginalJavaProjectileMain{}, it relies on a few methods
calculating the related ``instance models'' on its behalf. Those methods are a
specific \textit{interpretation}\footnote{Or ``view.''} of the theory knowledge.
Specifically, it is a \textit{calculation-focused} interpretation. For example,
to calculate \(p_\text{land}\), \refOriginalJavaProjectileMain{} relies on the
method \inlineCode{java}{func_p_land} to calculate the landing position.
\refOriginalJavaProjectilePLandMethod{} is one possible exported method for
calculating \(p_\text{land}\) given \(v_\text{launch}\), \(\theta\), and
\(\mathbf{g}\).

\originalJavaProjectilePLandMethod{}

Of course, to generate this method in \refOriginalJavaProjectilePLandMethod{},
Drasil relies on a sufficient capture of its underlying knowledge, and a means
of transforming said knowledge to ``code'' (e.g., sufficient information that
answers: What do we want to define? How can it be converted to code?). This
capture of theories is done using Drasil's representation of an ``Instance
Model.'' To build it, Drasil uses information gathered from what users fed in
about it, via \refOriginalLandPosTheoryDefinition{}.

\originalLandPosTheoryDefinition{}

Now, let's unpack \refOriginalLandPosTheoryDefinition{}. \inlineHs{landPosIM} is
an instance of an \InstanceModel{}\footnote{Drasil's encoding of the ``Instance
    Models.''}, containing the equational definition component
(\inlineHs{landPosRC}), and meta-level information about the theory, including
constraint ranges, notes, and a derivation\footnote{Mostly omitted here for the
    sake of conserving space.}.

To transform \refOriginalLandPosTheoryDefinition{} into
\refOriginalJavaProjectilePLandMethod{}, \textit{interpretation}\footnote{More
    accurately, this interpretation is ``domain-specific interpretation''
    \cite{Czarnecki2005}.} occurs on \inlineHs{landPosRC}
(\refOriginalRelToQDHaskell{}).

\originalRelToQDHaskell{}

\refOriginalRelToQDHaskell{} shows the definition of \relToQD{}, a function that
attempts to convert arbitrary relations into coherent ``quantity
definitions''\footnote{Or, ``variable definitions.''} (\QDefinition{}), which
Drasil's ``\acs{srs} to code'' generator relies on. \inlineHs{landPosRC} is an
instance of a \RelationConcept{} (\refOriginalRelationConcept{}), a coupled
mathematical relation (encoded using \Relation{}s\footnote{\Relation{} is
    Drasil's \acs{dsl} encoding arbitrary mathematical relations. We will talk more
    about this later. For now, this is enough.}), natural language description of
the relation, and descriptive name.

\originalRelationConcept{}

\relToQD{} is used on \textit{all} of the captured \InstanceModel{}s of a case
study to find code it can generate. However, these \InstanceModel{}s carry
arbitrary \Relation{}s, which are arbitrary mathematical relations. As such, an
issue arises: since \relToQD{} only works with ``variable defining''
relations\footnote{In some sense, calling these relations ``variable defining''
    is also an issue of itself. We're overloading \(=\) to mean definition.}, we're
limited to only one \textit{kind} of theory. Even with this restriction, we're
further limited to only a specific formation of that theory (e.g., relations of
the form \(x = f(a, b, c, \ldots{})\))!

\subsection{Problems}
\label{chap:modelkinds:sec:transforming-theories-to-code:subsec:problems}

In the sense that we really want to be able to use \relToQD{} (or something like
it) to transform arbitrary well-understood \textit{theories} into code fragments
for Drasil's code generator, it has at least 3 problems:
\begin{enumerate}
    \item[\namedlabel{mk:issue:1}{P1}] it only handles one theory kind:
        variable definitions,
    \item[\namedlabel{mk:issue:2}{P2}] and for those definitions, it requires a
        specific form, thereby limiting users to very specific usage, views, and
        transcription,
    \item[\namedlabel{mk:issue:3}{P3}] and it implicitly assumes that all inputs
        will be of this theory kind, or else it causes a panic.
\end{enumerate}

As a result of \ref{mk:issue:1}, we aren't able to encode adequately all the
theories we're interested in using, and want to generate representational code
of. In particular, as Drasil is heavily guided by physics-focused case studies,
\acsp{ode} are desired! When we want to use \acsp{ode} in the solution of a
problem, extra information is required. For example, we might need to give
Drasil (and/or developers) information about a desired approximation formula
with particular ``settings.'' Drasil does circumvent this issue for \acsp{ode},
but we would like to reconcile the half-measures and push all necessary
information back in to the theory encodings.

Assuming we wanted to describe the theory of a line, there are many ways we can
describe the equation: polynomial (\(ax + by + c = 0\)), slope-intercept form
(\(y = mx + b\)), point-slope form (\(y_1 - y_2 = m(x_1 - x_2)\)), and so on.
However, as a result of \ref{mk:issue:2}, we are forced to use the ``simple''
slope-intercept form, even though we are aware of other forms and may prefer to
describe it in other forms.

As a result of \ref{mk:issue:3}, when Drasil's users are encoding theories in
Drasil, they might be misled to think that any of their encoded theories is
fully-understood to Drasil and actually usable as part of the generated solution.

Together, these issues arise because of a lack of sufficient \textit{depth} and
\textit{breadth} in the contained knowledge of the theory encodings. Because the
theory encodings rely on ``flat'' information (e.g., the relations),
transforming them programmatically is challenging. Of course, if two programmers
were to read this information in the \acs{srs}, they might be able to use it.
However, the programmers understand the context of the theory, and are able to
\textit{recognize} (from any form of a theory) if and how it can be transcribed
as a computation. Now, imagine if we wanted to  So, how can we mitigate all of
these issues?

Just as we may discuss the specifics of implementing any particular theory in
our manual implementation of a software artifact, we assume prior learned
knowledge about mathematical expressions, such as which ones we know we can
somehow translate into code. To mitigate these issues, we must further capture
this background knowledge because raw relations carry too little information
about how to transform into coded. So, now, more concretely, we ask: how could
we have avoided these individual problems?

\begin{longtable}[c]{>{\raggedright}p{0.2\linewidth}>{\raggedright\arraybackslash}p{0.65\linewidth}}
    \textit{To avoid~\ldots{}} & \textit{we needed~\ldots{}}                                    \\
    \ref{mk:issue:1}           & knowledge about more \textit{kinds} of theories,               \\
    \ref{mk:issue:2}           & to decompose the relations into its set of
    logical components and capture information about how instances can be
    transformed into various forms,                                                             \\
    \ref{mk:issue:3}           & a signifier for each theory \textit{kind} we're interested in. \\
\end{longtable}

Succinctly, this means we need a system for \textit{classifying} and
capturing/exposing the \textit{structure} of more \textit{kinds} of theories.
This also means \RelationConcept{}s and \Relation{}s are insufficient couriers
of ``theory knowledge.'' In particular, we should be careful to ensure that the
abstractions are capable of reproducing the original knowledge we abstracted
them out of\footnote{This also should typically create opportunity for more
    kinds of interpretation of the abstracted knowledge too.}. In other words, we
should ensure that we can re-create the original \RelationConcept{}s from the
newly created, structured, theories.

\section{Classifying Theories}
\label{chap:modelkinds:sec:classifying-theories}

As a result of the inability to enter in complex theories not of the form \(x =
f(a, b, c, \ldots{})\) and the desire to encode theories with
\acsp{ode}\footnote{As Drasil's examples are physics-focused and \acsp{ode} are
    common in physics problems, the ability to generate software that can solve
    \acsp{ode} is desired.}, Drasil's existing \acsp{ode} has a bit of temporary code
to circumvent the ``1 Theory Kind'' restriction. However, we won't focus on
\acsp{ode} in this work. Since we only have one kind of theory that is processed
down the ``normal'' route (\relToQD{}) along the path of code generation, we
will focus on it.

Continuing to use \(p_\text{land}\) as an example, recall
\refOriginalLandPosTheoryDefinition{}, as it is similar to any one of the other
theories in Drasil used for code generation (other than \acsp{ode}!). The code
generator currently understands how to convert ``quantity definitions''
(\QDefinition{}s, \refOriginalQDefinitionHaskell{}) into various snippets of
usable code that it can use to generate a whole software artifact.

\originalQDefinitionHaskell{}

\QDefinition{}s are ``higher-level'' versions of the existing theories. A
\QDefinition{} breaks up an equational theory (i.e., theories of the form \(x =
f(a, b, c, \ldots{}\)) into its logical constituents:
\begin{enumerate}
    \item a variable to be defined,
    \item an expression (formula) defining the variable,
    \item and a natural language explanation of the coupling.
\end{enumerate}

For our current needs, this is a sufficient interpretation of ``equational
theories.'' Additionally, since all existing instance models are processed by
\relToQD{}, we similarly have a clear path to converting the existing
\RelationConcept{}s into \QDefinition{}s where applicable. Specifically, to
convert them, we merely need to hand-process \relToQD{} in the case studies,
where we normally write \RelationConcept{}s\footnote{Note: since the
    \QDefinition{}s are merely unpacked copies of the existing \RelationConcept{}s
    used in code generation, we can also similarly re-create those
    \RelationConcept{}s we're aiming to replace. In other words, we can also make
    \RelationConcept{}s a ``view'' of \QDefinition{} by re-interpreting them while
    connecting the formula to the defined variable with an equality.}. For example,
if we wanted to convert \inlineHs{landPosRC} into a \QDefinition, we might do it
as \refCurrentLandPosRCtoQD{}, where \inlineHs{E.landPosExpr} is a variable
containing the expression (formula) defining the \inlineHs{landPos} symbol.

\currentLandPosRCtoQD{}

However, merely replacing all \RelationConcept{}s with \QDefinition{}s is also
limiting. We want to use more \textit{kinds} of theories too! Thus,
\ModelKinds{} \footnote{\ModelKinds{} is based on a prototype by Dr.\ Jacques
    Carette. Shortly after implementing it, the Drasil Research Team discussed
    changing the name of ``models'' to ``theories,'' to which, we did, but have not
    propagated through the Drasil codebase, pending analysis on how it might affect
    the \acs{srs} template \cite{DrasilIssue2599RenamingModels}. Thus,
    ``\ModelKinds{}'' might change to ``\TheoryKinds{}'' or the singular
    ``\TheoryKind{}.''} (\refOriginalNewModelKindsHaskell{}) is created and
propagated through the various theory encodings to replace \RelationConcept{}s
(and \Relation{}s).

\originalNewModelKindsHaskell{}

\ModelKinds{} is an \ACF{adt}, enumerating over all currently
known\footnote{Criteria for ``currently known'' being ``ones this work has
discussed up until this point.''} \textit{kinds} of theories in Drasil. The
constructors of \ModelKinds{} are intended to carry structured encodings of
theory knowledge. By switching to \ModelKinds{}, theories can be created using
the structural information of theories, and the theories using structured
encodings are easier to refine and interpret\footnote{Such as for code
generation!} than the unstructured copies. At the moment, \ModelKinds{} has 3
kinds of ``models'' (theories) that can be created:

\begin{enumerate}
    \item \EquationalModel{}s, for theories that define a variable with some
          expression
    \item \DEModel{}s, for theories involve \acsp{ode},
    \item and \OthModel{}s, for everything else.
\end{enumerate}

Now, with \EquationalModel{}, we may finally rebuild \inlineHs{landPosIM}
(\refOriginalLandPosTheoryDefinition{}) using the improved structure
(\refCurrentLandPosIMQD{}). At the moment, it does not change the expected
generated artifacts, however, if desired, one can change output types of
theories in the \acs{srs} (or code) by adding new kinds of \QDefinition{} and
\ModelKinds{} interpretations.

\currentLandPosIMQD{}

Continuing, the \DEModel{}s still require a bit of manual circumvention, but
bringing some of the information back into the core ``theories'' to use in the
intended flow of knowledge, should be possible \cite{Chen2022MEng}. Drasil also
currently has many more kinds of theories, not all usable in code generation
\textemdash{} more than just equational models! However, we won't focus on them
\textit{quite yet}, we'll get back to this \Cref{chap:more-theory-kinds}.

Finally, as a result of incorporating \ModelKinds{} in Drasil, we:
\begin{enumerate}[label={(\alph*)}]
    \item add a system of classifying theories (through type information and
          constructors),
    \item change how users enter theories in Drasil (by making them enter in the
          structural components into ``cookie cutter'' shapes),
    \item gain information about theory capabilities (based on type
          information).
    \item and, ultimately, don't need \relToQD{} anymore.
\end{enumerate}

Now, on the note of exposing theory capabilities, we mean that we know have
static type information, at the level of the whole Haskell code compilation,
about how theories can be used. For example, if we previously had a function
that converts \RelationConcept{}s into pretty typeset boxes in \LaTeX{}, we
likely would have had similar troubles (as discussed above) in adding an
alternative printing form of the theory. However, now, we might, for example,
create a sub-kind of \EquationalModel{}s that are for equations of lines. Then
we can re-create the printing function for them, adding a parameter for the
printing form. The printing function now is able to more easily make use of the
logical constituents of a theory.

More important than printing to textual artifacts, we are interested in
generating code. Specifically, we are interested in knowing which
\EquationalModel{}s are usable in code generation, based on their defining
expressions.
