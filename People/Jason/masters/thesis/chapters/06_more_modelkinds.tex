\chapter{More Theory Kinds}
\label{chap:more-theory-kinds}

\begin{writingdirectives}
	\item What other theory kinds can we find in the existing Drasil examples?
	\item How is ModelKinds affected by the language division in
	\Cref{chap:lang-division}?
	\item What theories currently are used in Drasil?
\end{writingdirectives}

In this chapter, we return to examining more of the existing theories in Drasil,
and create alternative, structured versions of them to create further
opportunities for domain-specific interpretation.

\section{Remaining Theories}
\label{chap:more-theory-kinds:sec:remaining-theories}

As of writing, there are mainly two kinds of theories that are used in code
generation: the equational model and the \acs{ode} models. However, they aren't
the only kinds of theories encoded in Drasil. There are at least two other
re-occuring kinds of theories. One of them is a theory that shows multiple ways
to define a quantity. For example, \refExampleEquationalRealmImg{} is an example
from the \acs{dblpendulum} case study. Meanwhile, the other theory imposes
constraints on a system. For example, \refExampleEquationalConstraintsImg{} is
an example from the \acs{ssp} case study.

\refExampleEquationalRealmImg{} shows an abstract ``force'' variable,
\(\mathbf{F}\), which be defined by at least two different expressions: \(m
\mathbf{a}\) or \(-{\mathbf{T}_{1}}
\sin\left({\theta_{1}}\right)+{\mathbf{T}_{2}} \sin\left({\theta_{2}}\right)\).
This theory, in particular, is used abstractly, providing reasoning for building
different ways to define \(\mathbf{F}\), but also creating opportunity for
further specialization in other theories or systems. For example, this theory is
used as part of the derivation of another theory
(\refExampleEquationalRealmUsageImg{}).

Meanwhile, \refExampleEquationalConstraintsImg{} shows a theory that explains
that the conditions under which a body may be considered to be in static
equilibrium. This theory was intended to be used as part of constricting other
theories. For example, it was used as part of the development of another theory
(\refExampleEquationalConstraintsUsageImg{}).

\exampleEquationalRealmImg{}

\exampleEquationalConstraintsImg{}

Finally, with these theories in mind, we may start to add structured containers
to replace the existing \RelationConcept{}s and \Relation{}s, currently used to
build them, in hopes that we can eventually learn how we really want to use
these theories.

\section{\textquotedblleft{}Classify All The Theories\textquotedblright{}}
\label{chap:more-theory-kinds:sec:classify-all-the-theories}

\subsection{Equational Realms}
\label{chap:more-theory-kinds:sec:classify-all-the-theories:subsec:equational-realms}

Equational realms represent ``realms'' \cite{Carette2014realms}, sets of unique
axioms that are equivalently interpretable, focused on different ways to define
a particular variable. They may be specialized to become equational models.
\EquationalRealm{}s represent ``equational realm'' theories in Drasil and are
effectively \MultiDefn{}s. For example, we may define a theory with multiple
ways to define the horizontal force on an object:
\refCurrentExampleEquationalRealmHaskell{}.

\currentExampleEquationalRealmHaskell{}

\currentMultiDefnHaskell{}

\currentDefiningExprHaskell{}

\subsection{Equational Constraints}
\label{chap:more-theory-kinds:sec:classify-all-the-theories:subsec:equational-constraints}

``Equational constraints'' are theories that assert certain properties over
other theories. They use \ConstraintSet{}s under the hood
(\refCurrentConstraintSetHaskell{}) to hold a list of relations for assertion.

\currentExampleEquationalConstraintsHaskell{}

\currentConstraintSetHaskell{}

\subsection{Differential Equations}
\label{chap:more-theory-kinds:sec:classify-all-the-theories:subsec:differential-equations}

The capture of differential equations in Drasil is an active area of research.
Dong Chen continued work here, creating \NewDEModel{} \cite{Chen2022MEng} (a new
constructor in \ModelKinds{}) to start capturing information about linear
\acs{ode} systems. \DEModel{} is left as a temporary carriage for the remaining
theories to be similarly analyzed and re-built with a deeper depth of knowledge
capture, so that we can make better use of the information in them. Thanks to
Dong's work, Drasil is now able to generate software for the \acs{dblpendulum}
case study in Java, Python, C/C$++$, and C\# \cite{Chen2022MEng}. As such, this
research already has some success in enabling more theories to be encoded in
Drasil and appropriately used for various purposes.

\subsection{Theories Left Undiscussed}
\label{chap:more-theory-kinds:sec:classify-all-the-theories:subsec:theories-left-undiscussed}

While we have analyzed a few theories and how they're used, there are still many
theories left undiscussed\footnote{\ModelKinds{} is an incomplete enumeration.}.
Following Drasil's methodology, they will only be analyzed and captured as
necessary. Notably, \ModelKinds{} still contains \OthModel{}, meaning that there
still exist theories in Drasil, which are used in justifications, of which we
haven't yet decided how we want to use yet\footnote{This is ``future work'' for
now.}.

Finally, after \Cref{chap:lang-division} and this brief structuring of some
existing theories, we end with \ModelKinds{} appearing as in
\refCurrentModelKindsHaskell{}\footnote{In the \refCurrentModelKindsHaskell{}
definition, there are two (2) TO-DO notes that you may disregard. The first one
is merely a note for analyzing ``well-understood'' copies of our existing
\acsp{ode}, and the second one refers to models that haven't yet been fully
analyzed for how they will be used (other than for display).}.

\currentModelKindsHaskell{}

Finally, as a result of implementing \ModelKinds{} (\Cref{chap:modelkinds}) and
the expression language division (\Cref{chap:lang-division}), we are now able
to, in at least one way, restrict the terms we use in different kinds of
theories across different contexts. Ultimately, this adds some assurance that
all generated artifacts only contain relevant language in them, because we have
filtered out terms by their context. However, we have yet to discuss
\textit{well-typedness} of expressions.
