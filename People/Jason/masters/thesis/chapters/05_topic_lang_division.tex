\chapter{Expression Language Division}
\label{chap:lang-division}

\begin{writingdirectives}
    \item What are Relations really?
    \item What is \Expr{}?
    \item How is \Expr{} used?
    \item How can we restrict \Expr{} terms to particular areas?
\end{writingdirectives}

In this chapter, we will discuss how Drasil captures relations and general
mathematical expressions, and the issues associated with the single universal
language approach Drasil takes.

\section{Relations}
\label{chap:lang-division:sec:relations}

In \Cref{chap:modelkinds}, we discussed how Drasil captures theories.
Specifically, we discussed how Drasil heavily relies on relations for defining
them. However, we never discussed the relation encoding: \Relation{}.

\originalRelation{}

\Relation{} (\refOriginalRelation{}) is actually a type alias for \Expr{}
(\refOriginalExprHaskell{}), which captures general mathematical expressions. As
such, \Relation{}s aren't really ``relations.'' In fact, \Relation{}s can take
on any mathematical expression, such as \(3x+2\), which is obviously not a
``relation.''

\originalExprHaskell{}

\section{A Mathematical Language}
\label{chap:lang-division:sec:a-mathematical-language}

\Expr{} (\refOriginalExprHaskell{}) is an \acs{adt}\footnote{In \acs{gadt}
syntax.}, representing the \ACF{ast} of a mathematical expression language. It
was grown following the needs of encoding the case studies in Drasil. As such,
it contains a wide range of mathematical terms, including variables, numerics,
derivatives, common mathematical operations, and function applications. In
addition to the example expression (a relation) from
\refOriginalLandPosTheoryDefinition{}, if, for example, we wanted to transcribe
\(3 \tan{}(x) = y\) in Drasil, we might write \refPseudoExampleExpression{}.

\pseudoExampleExpression{}

Of course, this is great! We're able to transcribe any expression contained in
the language of \Expr{}. \refPseudoExampleExpression{} is a great example which
we can confidently re-write as an equational model (as in
\Cref{chap:modelkinds}):

\begin{pseudohaskell}{Example Expression in a QDefinition}{pseudoExampleExpressionAsQDefinition}
theQDef :: QDefinition
theQDef = mkQuantDef y (dbl 3 $* tan (sy x))
\end{pseudohaskell}

With \Cref{lst:pseudoExampleExpressionAsQDefinition}, we can create an instance
model containing an equational model as part of the solution of the greater
problem. One of the important requirements of instance models is that we expect
them to be unambiguously translatable into some component of the solution, up to
user choices (such as choosing what ``code'' representation of integers and
reals to use). This is where \Expr{} shows that it is a bit of a double-edged
sword. For example, we can re-write the expression used in
\Cref{lst:pseudoExampleExpressionAsQDefinition} using a trigonometric integral
(e.g., using \(\tan{}(x) + C = \int \frac{1}{\cos^2(x)}dx\) \cite{Math2Org}, we
obtain \(y = \int \frac{3}{\cos^2(x)}dx\) using \(C = 0\),
\Cref{lst:pseudoExampleExpressionAsQDefinitionBad}), but then the transformation
into a naive set of steps to follow isn't as simple.

\begin{pseudohaskell}{Example Bad Expression in a QDefinition}{pseudoExampleExpressionAsQDefinitionBad}
theQDef :: QDefinition
theQDef = mkQuantDef y (integAll (eqSymb x) (dbl 3 $/ (cos (x) $^ int 2)))
\end{pseudohaskell}

With \Cref{lst:pseudoExampleExpressionAsQDefinitionBad}, which solution to use
isn't clear. Should Drasil generate something to solve the integral? Should
Drasil replace the integral with the original $\tan$? How would it recognize to
convert it into $\tan$ without running into a similar problem as \relToQD{}
(\Cref{chap:modelkinds})? Should we expect the target generated languages to
have functionality for solving any integral we give it?

\textit{We} shouldn't be the ones answering any of these questions\footnote{We
should be imposing the least possible amount of things on users!}. Rather, we
should be deferring these questions to users, giving them options for choosing
answers to these questions. In other words, the transformation from \Expr{}s to
``code'' is not naively \textit{total}, which is problematic. However, these
questions are also still too complex to be external to the \acs{srs}, and as
such, we would need this theory to be a higher-level theory, which the user
refines with information that would appear in the \acs{srs} for domain-experts
to audit. In other words, we don't want integrals, like in this example, to
appear in the concrete theories (instance models) that are used in code
generation, at least without extra information about how to solve them embedded
closely.

Additionally, \Expr{} contains functionality for ``actors'' and ``messages,''
concepts in the world of \acs{oo} programming. While these are helpful for the
translation of mathematical expressions that might use some functionality from
``actors,'' users building an \acs{srs} should be relatively removed from the
actual programming. In other words, these terms are inappropriate for usage in
the \acs{srs}.

As such, we need to be able to restrict expression terms to their appropriate
contexts. At the moment, we lack information about when instances of expressions
are actually usable in code generation. As such, we're limited to relying on
implicit assumptions about this, and causing Drasil-panics when the assumption
is broken. To resolve this, we need to make the information \textit{explicit} in
Drasil, and use that explicit information appropriately. However, before we can
make it explicit, we need to understand the contexts in which Drasil needs
mathematical expressions and how they are used in those contexts.

Namely, there are at least three focal areas of interest in Drasil where we use
mathematical expressions:

\begin{enumerate}

    \item In the \acs{srs} for abstract theories, where we discuss abstract
          concepts that are ultimately used in the derivations of concrete
          theories,

    \item In the \acs{srs} for concrete theories (such as instance models),
          where we expect them to be \textit{concrete} in a sense that they are
          trivially usable as part of a solution to an outlined problem,

    \item In generated artifacts, where we first need to convert the
          concrete theories into an intermediate representation of the generated
          artifacts and making any necessary specializations for the target
          language, before finally outputting them.

\end{enumerate}

\section{Splitting}
\label{chap:lang-division:sec:splitting}

To restrict terms to their appropriate contexts, we need to somehow know when
users are using the ``wrong'' terms. Since Drasil is written in Haskell, we can
use Haskell's type system to indicate this with relative ease. Namely, we can
\textit{split} up \Expr{}. Thankfully, the three areas of interest share a
subset of expressions: the mathematical expressions usable in concrete theories.
As such, we would like to divide the language using this language as a base for
the other two (\refLanguageDivision{}).

\languageDivision{}

After the work, we end up with three languages:

\begin{enumerate}

    \item \Expr{}, which is the subset of the original \Expr{} where all terms
          have definite values (e.g., literals, common unary and binary
          operations, symbols, and first-order function applications), and
          should be the language of the \QDefinition{}s usable in instance
          models (\refCurrentExprHaskell{}),
    
    \item \CodeExpr{}, which is expected to be related to a \textit{total}
          transformation function that re-writes terms in \CodeExpr{} into
          \acs{gool} for final rendering in a final software artifact
          (\refCurrentCodeExprHaskell{}),

    \item  \ModelExpr{}, which is approximately the \Expr{} with nearly
          everything from the original \Expr{}, except without code-oriented
          terms\footnote{As they should not be discussed in the \acs{srs}
          documents at all!} (\refCurrentModelExprHaskell{}). Notably,
          \ModelExpr{} contains all indefinite valued terms that were cut out of
          \Expr{}, such as derivatives, integrals, quantification, types as
          values (spaces), and continuous ranges for summations and products.

\end{enumerate}

Now we have 3 \acsp{adt} available for different kinds of expression languages
with ``mathematics'' as a common tie. One issue with splitting is that we now
force users of the languages to make a conscious choice about which language
they are using, and which language they need to import into their working files
respectively. This can become quite frustrating given the amount of overlap.
Thus, to alleviate the stress involved with writing out the same expression in
different languages, we use a \ACF{ttf}~\cite{Carette2009} encoding of the smart
constructors to build expressions. Using \Expr{} has a ``base'' shared language
between \ModelExpr{} and \CodeExpr{}, we can write \acs{ttf} encoding for them
individually that extends on \Expr{}'s functionality
(\refCurrentExprTTFHaskell{}, \refCurrentModelExprTTFHaskell{},
\refCurrentCodeExprTTFHaskell{}). The \acs{ttf} encoding allows us to seamlessly
write expressions in any of the languages at the same time, allowing type
variability along the type constraints of any expression. The typeclasses used
in the languages may be used to create constriants on the language used.

While \ModelExpr{}s \acs{ttf} encoding strictly contains the terms unique to
\ModelExpr{}, it is possible (and typical) to convert \Expr{}s into
\ModelExpr{}s for usage in generating the \acs{srs} documents. Additionally,
since many theories need to be representable as a single \Relation{}, we may
create a typeclass to create enforcement (\refCurrentExpressHaskell{}). By
instantiating it for various types, we are explaining how those terms can be
interpreted as a \ModelExpr{}.

\currentExpressHaskell{}
 
\section{Back to Theories}
\label{chap:lang-division:sec:back-to-theories}

Now that we've split up the expression languages, we may restrict their usage to
appropriate theory contexts (in \ModelKinds{} as per \Cref{chap:modelkinds}).
Approximately, we can ``upgrade'' \ModelKinds{} according to
\refPseudoPartialModelKindsUpgrade{}\footnote{Note that we also need to slightly
update \QDefinition{}s: \refCurrentQDefinitionHaskell{} to create a type
parameter for the used expression type.}, creating a type argument in
\ModelKinds{} so that instance models can be restricted to carry ``\ModelKinds{}
\Expr{}'' while the other theory types may carry ``\ModelKinds{}
\ModelExpr{}''\footnote{However, using \Expr{} and \ModelExpr{} is not quite
accurately representative of ``usable in code'' versus ``not,'' which we are
going to fix \cite{DrasilIssue2853AlternativeModelKinds}.}. This allows us to
ensure that all expressions written in Drasil for instance models have an
semantic counterpart that they can be translated into for code generation in
various \acs{oo} languages. Furthermore, now we may return to \ModelKinds{} and
continue uncovering the various \textit{kinds} of theories currently discussed
in Drasil's case studies.

\pseudoPartialModelKindsUpgrade{}
