\documentclass{article}

\usepackage{amsmath, mathtools}
\usepackage{amsfonts}
\usepackage{amssymb}
\usepackage{graphicx}
\graphicspath{{../assets}} % Bring in the assets

\usepackage{colortbl}
\usepackage{xr}
\usepackage{longtable}
\usepackage{xfrac}
\usepackage{tabularx}
\usepackage{float}
\usepackage{siunitx}
\usepackage{booktabs}
\usepackage{multirow}
\usepackage{caption}
\usepackage{pdflscape}
\usepackage{afterpage}
\usepackage[dvipsnames,table,xcdraw]{xcolor}

\usepackage[normalem]{ulem}

\usepackage{hyperref}
\hypersetup{ colorlinks, citecolor=blue, filecolor=black, linkcolor=red,
    urlcolor=blue }
\usepackage{cleveref}

\usepackage{tikz}
\usepackage{varwidth}
\usetikzlibrary{shapes,arrows,cd,babel,arrows.meta,graphs,graphdrawing}
\usegdlibrary {layered}

% Credit to Gabriel Devenyi for this bibliography cfg:
% github.com/gdevenyi/mcmaster.latex
\usepackage[ style=numeric-comp, backend=biber, sorting=none, backref=true,
  maxnames=99, alldates=iso, seconds=true ]{biblatex} % bibliography
\addbibresource{../references.bib}

%\usepackage{refcheck}


% Force todonotes to play nicely in other floats: credits to
% https://tex.stackexchange.com/a/402075
\usepackage{marginnote}
\renewcommand{\marginpar}{\marginnote}

%% Comments

\usepackage{color}

\newif\ifcomments\commentstrue %displays comments
%\newif\ifcomments\commentsfalse %so that comments do not display

% Changes: Use ``todonotes'' package for the comments, but don't use it for
% inlined template explanation information.

\ifcomments
    \newcommand{\plt}[1]{\textcolor{magenta}{[TPLT \textemdash{} #1]}}
    \usepackage[backgroundcolor=yellow,colorinlistoftodos]{todonotes}
    \setlength{\marginparwidth}{0.75in}
    \reversemarginpar % place on left-hand side
\else
    \newcommand{\plt}[1]{}
    \usepackage[disable]{todonotes}
\fi

\newcommand{\wss}[1]{\todo[color=blue]{SS: #1}}
\newcommand{\an}[1]{\todo[color=cyan]{\authinitials{}: #1}}

\newcounter{needscitationcounter}
\newcommand{\needscitation}{\stepcounter{needscitationcounter}\todo[color=red]{NC \theneedscitationcounter{}: Needs citation.}}
\newcommand{\nc}{\needscitation}

%% Common Parts

\newcommand{\progname}{BeamBending}
\newcommand{\authname}{\small{\textit{Team Drasil}}\\Jason Balaci}
\newcommand{\authinitials}{JB}

\usepackage{hyperref}
\hypersetup{colorlinks=true, linkcolor=blue, citecolor=blue, filecolor=blue,
            urlcolor=blue, unicode=false}
\urlstyle{same}

\usepackage{newunicodechar}

\newunicodechar{Ȧ}{$\dot{A}$}


\title{Reflection Report on \progname{}}
\author{\authname}

\begin{document}

\maketitle

\section{Changes in Response to Feedback}

Please note that all feedback is recorded on the
\href{https://github.com/balacij/Drasil/issues}{issue tracker} used to manage my
project. I will only be discussing a select few things here.

\subsection{SRS \sout{and Hazard Analysis}}

\textit{Note: we did not do a Hazard Analysis, and so I've striked it out of the
    subsection title.}

The original structure of my SRS remained in tact after feedback primarily
because I had Dr.~Smith's assistance (and textbook~\cite{BeerJohnston1981}) to
give me advice from the start.

The majority of the feedback related to my SRS noted minor mistakes, (my)
misunderstanding of the intent behind some sections, or issues related to
theories and their explanations. The explanations of my theories still need
work, but I feel a lot more comfortable with them. Dr.~Smith's feedback along
with Sam and others has cleaned up my document significantly. A notable change
is the Software and Physical constraints related to my inputs (where I
previously had none) and Dr.~Smith suggested concrete ones with ample
justification.

The feedback given to me during the presentations was especially beneficial at
the start, when I was still working on my SRS. I'm very thankful that I had the
very first SRS presentation, which led to fruitful discourse and concrete advice
that allowed me to continue to write my SRS. I think that taking the advice
initially given to me, allowed me to later receive less problematic feedback
later.

\subsection{Design and Design Documentation}

Since my software is relatively simple, the design and design documentation
related to my prototype was minimal and there was no real feedback that needed
to be addressed (other than using realistic sizes of forces).

Related to the Drasil implementation of my SRS, the feedback is ongoing, and I'm
continuing to try to get my project working fully with code generation.

\subsection{VnV Plan \sout{and Report}}

\textit{Note: Sam and I did not do a VnV Report, and so I've striked it out of
    the subsection title.}

Other than general advice on cleaning up my VnV Plan, fixing grammar and typos,
adjusting numbers, and other minor issues, the nontrivial feedback focused on
clarifying tests (specifically those non-functional) where I deferred them to
Drasil and the Drasil research team. I tried my best to clarify why, but I
believe that I will need to discuss with the Drasil team further about
\textit{how} and \textit{why} I felt it was appropriate to defer the onus of
testing to Drasil. If the Drasil team believes it was inappropriate/incorrect to
do so, then I will also ask about and explore how we can make it
appropriate/correct to do so.

\section{Design Iteration (LO11)}

Unfortunately, the software has not yet been created. The prototype is too
simplistic (and reductive) to be notable to discuss here\footnote{It is a single
file that calculates the solution to the deflection problem, but it does no
validation on inputs and does not adequately format outputs.}. Furthermore, the
design of the solution program is mostly decided by Drasil (up to user
configuration options).

\section{Design Decisions (LO12)}
\label{design-decisions}

When we started the project, our scope was slightly larger: focusing on building
solvers for general distributed load application functions. However, we quickly
had to shift into focusing on only one kind, with a numerical solution as
opposed to analytical: third-order polynomials. Since we do not generate
software yet, we will need to revisit this section when we can.

One other notable constraint is that I did not want to compromise the quality of
the SRS (nor background mathematics) in favour of code generation when
converting the SRS into the Drasil-encoded form. This constraint restricts my
usage of Drasil a bit, specifically with regard to solving differential
equations, where Drasil's code generator expects the SRS to mostly provide the
solution design in a readily digestible format\footnote{For example,
``functions'' are expected to be represented by vectors to capture sampling
points across a numerically-solved differential equation.}. This also means I
was using functionality that wasn't fully flushed out yet in Drasil (function
definitions, vectors, and manually-written code snippets).

The remainder of the ``design decisions'' are deferred to Drasil which has its
own ready-made program design to solutions for SRS-captured problems.

\section{Economic Considerations (LO23)}

I believe the market for my project would primarily be educational. The online
educational scene has plenty of free and open source resources for typical beam
configurations taught in an undergraduate degree. One notable difference is that
we \textit{technically} support general loading functions but are constrained by
the ability of Drasil to generate programs with complex input terms (here,
general mathematical expressions). Should Drasil be able to read in general
expressions, then the program might have a relatively uncommon feature in free
and open source programs. 

Since the program is specifically intended for educational purposes, it likely
won't have a market on its own. However, the impact (along with the other case
studies) on Drasil's market impact might be fruitful for Drasil. If, for
example, a company were interested in using Drasil to analyze beams or other
various architectural or engineering-focused tasks, then Drasil (or at least the
Drasil ethos) might be able to change current industry development practices.

Drasil itself could be something that is ``sold'' on a consulting or
service-as-needed basis. Companies involved with operating systems or managing
data centers for profit (Amazon, Apple, Microsoft, Red Hat, Canonical,
Cloudflare, and such) could benefit from having Drasil take a front seat in
their software development stack. Using Drasil (or the Drasil ethos) should
allow them to make unparalleled positive impacts to their software quality and
even decrease their (current) variable and long-term expenses.

\section{Reflection on Project Management (LO24)}

\subsection{How Does Your Project Management Compare to Your Development Plan}

As I am the sole team member of my project, there were no team activities (only
personal). I have professional experience as a software developer and project
manager, and so I followed my normal strategy for developing
software\footnote{My strategy is fairly conventional: understand the background
problem domain, understand objectives, break down objectives into a series of
tiny ``calls to action,'' gradually break down the tickets in an adaptive manner
(focusing on easy tickets at times to get the ball rolling, or the normal
logical order to make sure that everything is ``good'' the first time around),
and revising objectives, work, and calls to action (with external reviewers!)
until I've achieved the objectives.}. Professionally, I prefer to use JIRA, but
GitHub's basic issue tooling sufficed for my project. Co-ordination with
reviewers and Dr.~Smith was quite smooth as our class was quite friendly (and so
in-person chats were frequent) and active on GitHub (and so online chats are
productive).

\subsection{What Went Well?}

I'm quite comfortable with all tools I used during this project, and so I was
comfortable with all project management efforts, processes, and technology used.
Thankfully, during this course, I was able to focus my time more on
understanding the SRS scheme that Dr.~Smith theorized. I'm confident it has made
me a better software developer. While it might be difficult to get hypothetical,
future companies I work with to adopt the SRS-first approach to software
development and the Drasil ethos, I am confident that I will enforce some sort
of usage of it. The SRS brings upon a coherent line of questioning and problem
reduction and exploration that I've never had the chance to do in the past.
Admittedly, quite a lot of my development experience feels chaotic in comparison
to the principled, document-driven approach that this class encourages us to
take.

\subsection{What Went Wrong?}

I don't believe there were any processes or technology that I had difficulty
with where I would claim that ``something went wrong.'' One might claim that the
difficulties I experienced with Drasil might go here, but I will push that to
the design section (\Cref{design-decisions}) since we (Dr.~Smith, Sam Crawford,
and I) anticipated problems. Thankfully, I was able to spend the majority of my
time without fiddling with tools to get past issues.

\subsection{What Would you Do Differently Next Time?}

This is my first time using a principled document-driven approach to software
development. The next time I will do it again will likely be just practice where
I can try to make further improvements to my work plan. However, before the
``next time,'' I will make sure to make suggest (or implement) some quality of
life improvements to Drasil to accommodate me. I've already made tickets on the
Drasil's issue tracker and I will likely be making more. One important thing I
would like to have for future authors is a ``guided tutorial for learning about
Drasil,'' where users are encouraged to create a local development environment
with Drasil, start the guided tutorial template, and use it as a basis for their
own projects.

In the future, I will definitely recommend colleagues to adopt the
document-driven approach to software development.

\end{document}
