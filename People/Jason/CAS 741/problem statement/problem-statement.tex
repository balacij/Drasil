\documentclass{article}

\usepackage{tabularx}
\usepackage{booktabs}

\title{Problem Statement and Goals\\\progname{}}

\author{\authname}

\date{}

%% Comments

\usepackage{color}

\newif\ifcomments\commentstrue %displays comments
%\newif\ifcomments\commentsfalse %so that comments do not display

\ifcomments
\newcommand{\authornote}[3]{\textcolor{#1}{[#3 ---#2]}}
\newcommand{\todo}[1]{\textcolor{red}{[TODO: #1]}}
\else
\newcommand{\authornote}[3]{}
\newcommand{\todo}[1]{}
\fi

\newcommand{\wss}[1]{\authornote{blue}{SS}{#1}} 
\newcommand{\plt}[1]{\authornote{magenta}{TPLT}{#1}} %For explanation of the template
\newcommand{\an}[1]{\authornote{cyan}{Author}{#1}}

%% Common Parts

\newcommand{\progname}{BeamBending}
\newcommand{\authname}{\small{\textit{Team Drasil}}\\Jason Balaci}
\newcommand{\authinitials}{JB}

\usepackage{hyperref}
\hypersetup{colorlinks=true, linkcolor=blue, citecolor=blue, filecolor=blue,
            urlcolor=blue, unicode=false}
\urlstyle{same}



% Configure font and file encodings, and language as Canadian English
\usepackage[T1]{fontenc}
\usepackage[utf8]{inputenc}
\usepackage[canadian]{babel}
\usepackage{lmodern}
\usepackage{anyfontsize}

% Required for biblatex, but also adds functionality for quotation
\usepackage{csquotes}

% Credit to Gabriel Devenyi for this bibliography cfg:
% github.com/gdevenyi/mcmaster.latex
\usepackage[
  style=numeric-comp,
  backend=biber,
  sorting=none,
  backref=true,
  maxnames=99,
  alldates=iso,
  seconds=true
]{biblatex} % bibliography
\addbibresource{references.bib}

\begin{document}

\maketitle

\begin{table}[hp]
    \caption{Revision History} \label{TblRevisionHistory}
    \begin{tabularx}{\textwidth}{llX}
        \toprule
        \textbf{Date} & \textbf{Developer(s)} & \textbf{Change}                       \\
        \midrule
        Jan. 20, 2023 & Jason                 & Template created \& committed to git. \\
        Jan. 20, 2023 & Jason                 & Template filled in.                   \\
        \bottomrule
    \end{tabularx}
\end{table}

\section{Problem Statement}
\label{problem-statement}

Beams safely support many constructions under load. Everything from bridges and
skyscrapers to commercial and residential properties use beams to safely carry
load by distributing the stress into their foundations and the ground\
\cite{Moscovitch2020}. Beams are flat, horizontal structural elements. They bear
load perpendicular to their horizon\ \cite{Moscovitch2020}.

Typically, inhabitants of residential constructions expect their floors to be
flat, balanced, and rigidly unmoving, or else they might feel uncomfortable in
their space. As such, beams must be rigidly fixed in place and be capable of
transferring all imposed loads down to the foundations of the buildings and the
ground. However, for other applications, such as bridges and beds of machine
tools, beams are free, within reason, to move horizontally\
\cite{BirdChivers1993}. \textit{Simply supported} beams are one kind of beams
that are commonly found in these other applications.

Simply supported beams use only two supports: a \textit{pinned} support and a
\textit{roller} support\ \cite{Lemonis2022}. The pinned support is fixed and
unmoving, while the roller support allows the beam to expand or contract axially
\cite{Lemonis2022}. To understand how simply supported beams handle uniformly
distributed loads, we aim to approximate the curve of deflection of a beam under
various conditions.

\subsection{Problem}
\label{problem-statement:problem}

Under simplified assumptions (i.e., Euler-Bernoulli beam theory
\cite{EulerBernoulliWiki}), to understand how various simply supported beams can
safely carry uniformly distributed loads, engineers must understand how beams
deflect and how the ends react.

\subsection{Inputs and Outputs}
\label{problem-statement:inputs-and-outputs}

As the scope of the work is relatively limited, the inputs and outputs are also
relatively limited. The inputs and outputs are as follows, following the
essential aspects of \cite{Lemonis2022}:

\subsubsection{Inputs}
\label{problem-statement:inputs-and-outputs:inputs}

\begin{enumerate}
    \item Beam structural properties:
        \begin{enumerate}
            \item length,
            \item material modulus of elasticity (Young's modulus), and
            \item moment of inertia.
        \end{enumerate}
    \item Magnitude of uniformly imposed force/load.
\end{enumerate}

\subsubsection{Outputs}

\begin{enumerate}
    \item At the supports, respectively:
        \begin{enumerate}
            \item force reactions, and
            \item angle of rotation.
        \end{enumerate}
    
    \item Related to the beam:
        \begin{enumerate}
            \item maximum deflection distance,
            \item bending moment, and
            \item transverse shear force.
        \end{enumerate}
\end{enumerate}

\subsection{Stakeholders}
\label{problem-statement:stakeholders}

As this work is a task of CAS\ $741$, the stakeholders of this case study
includes Dr.\ Spencer Smith (who will be reviewing the work), myself (Jason
Balaci, who will be taking authorship of the work and whose grade is dependent
on this work), and peers in CAS\ $741$ (notably, reviewers). Additionally, as
this work is expected to be performed using, and contribute to, Drasil\
\cite{Drasil2023}, Dr.\ Jacques Carette is another stakeholder in the final
product and contributions to Drasil. Notably, as a colleague in CAS\ 741 and in
the Drasil Research Team, Sam Crawford is also a stakeholder in this work.
Finally, as this work will additionally become a case study under Drasil, the
greater community related to Drasil is also a stakeholder.

\subsection{Environment}
\label{problem-statement:environment}

As this work will be built using Drasil, it will be made in Haskell according to
the Haskell2010 language specification\ \cite{Haskell2010}, and it is limited to
producing the software artifacts which Drasil is able to generate. Since Drasil
already covers generating software artifacts for general-purpose programming
languages that compile and run on major operating systems and Central Processing
Unit (CPU) architectures, usability of the final software artifacts should not
be an issue to the majority of users. However, should foreign hardware or
software be desired as an environment, an extension should be built in Drasil to
allow for generating suitable software artifacts.

\section{Goals}
\label{goals}

\section{Stretch Goals}
\label{goals:stretch-goals}

\section{Potential Changes}
\label{potential-changes}

\newpage

\printbibliography[heading=bibintoc]

\end{document}
