\documentclass{article}

\usepackage{tabularx}
\usepackage{booktabs}

\title{Problem Statement and Goals\\\progname{}}

\author{\authname}

\date{}

%% Comments

\usepackage{color}

\newif\ifcomments\commentstrue %displays comments
%\newif\ifcomments\commentsfalse %so that comments do not display

\ifcomments
\newcommand{\authornote}[3]{\textcolor{#1}{[#3 ---#2]}}
\newcommand{\todo}[1]{\textcolor{red}{[TODO: #1]}}
\else
\newcommand{\authornote}[3]{}
\newcommand{\todo}[1]{}
\fi

\newcommand{\wss}[1]{\authornote{blue}{SS}{#1}} 
\newcommand{\plt}[1]{\authornote{magenta}{TPLT}{#1}} %For explanation of the template
\newcommand{\an}[1]{\authornote{cyan}{Author}{#1}}

%% Common Parts

\newcommand{\progname}{BeamBending}
\newcommand{\authname}{\small{\textit{Team Drasil}}\\Jason Balaci}
\newcommand{\authinitials}{JB}

\usepackage{hyperref}
\hypersetup{colorlinks=true, linkcolor=blue, citecolor=blue, filecolor=blue,
            urlcolor=blue, unicode=false}
\urlstyle{same}



% Configure font and file encodings, and language as Canadian English
\usepackage[T1]{fontenc}
\usepackage[utf8]{inputenc}
\usepackage[canadian]{babel}
\usepackage{lmodern}
\usepackage{anyfontsize}

% Required for biblatex, but also adds functionality for quotation
\usepackage{csquotes}

% Credit to Gabriel Devenyi for this bibliography cfg:
% github.com/gdevenyi/mcmaster.latex
\usepackage[
  style=numeric-comp,
  backend=biber,
  sorting=none,
  backref=true,
  maxnames=99,
  alldates=iso,
  seconds=true
]{biblatex} % bibliography
\addbibresource{references.bib}

\begin{document}

\maketitle

\begin{table}[hp]
    \caption{Revision History} \label{TblRevisionHistory}
    \begin{tabularx}{\textwidth}{llX}
        \toprule
        \textbf{Date} & \textbf{Developer(s)} & \textbf{Change}                       \\
        \midrule
        Jan. 20, 2023 & Jason                 & Template created \& committed to git. \\
        Jan. 20, 2023 & Jason                 & Template filled in.                   \\
        \bottomrule
    \end{tabularx}
\end{table}

\section{Problem Statement}
\label{problem-statement}

Beams safely support many constructions under load. Everything from bridges and
skyscrapers to commercial and residential properties use beams to safely carry
load by distributing the stress into their foundations and the ground\
\cite{Moscovitch2020}. Beams are flat, horizontal structural elements. They bear
load perpendicular to their horizon\ \cite{Moscovitch2020}.

Typically, inhabitants of residential constructions expect their floors to be
flat, balanced, and rigidly unmoving, or else they might feel uncomfortable in
their space. As such, beams must be rigidly fixed in place and be capable of
transferring all imposed loads down to the foundations of the buildings and the
ground. However, for other applications, such as bridges and beds of machine
tools, beams are free, within reason, to move horizontally\
\cite{BirdChivers1993}. \textit{Simply-supported} beams are one kind of beams
that are commonly found in these other applications.

Simply-supported beams use only two supports: a \textit{pinned} support and a
\textit{roller} support\ \cite{Lemonis2022}. The pinned support is fixed and
unmoving, while the roller support allows the beam to expand or contract axially
\cite{Lemonis2022}.

\subsection{Problem}
\label{problem-statement:problem}

Under simplified assumptions (i.e., Euler-Bernoulli beam theory
\cite{EulerBernoulliWiki}), to understand how various simply-supported beams can
safely carry uniformly distributed loads, engineers must understand how beams
deflect and how the ends react.

\subsection{Inputs and Outputs}
\label{problem-statement:inputs-and-outputs}

\wss{Characterize the problem in terms of ``high level'' inputs and outputs.
    Use abstraction so that you can avoid details.}

\subsection{Stakeholders}
\label{problem-statement:stakeholders}

\subsection{Environment}
\label{problem-statement:environment}

\wss{Hardware and software}

\section{Goals}
\label{goals}

\section{Stretch Goals}
\label{goals:stretch-goals}

\section{Potential Changes}
\label{potential-changes}

\newpage

\printbibliography[heading=bibintoc]

\end{document}
