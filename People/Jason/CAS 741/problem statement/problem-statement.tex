\documentclass{article}

% Configure font and file encodings, and language as Canadian English
\usepackage[T1]{fontenc}
\usepackage[utf8]{inputenc}
\usepackage[canadian]{babel}
\usepackage{lmodern}
\usepackage{anyfontsize}

% Required for biblatex, but also adds functionality for quotation
\usepackage{csquotes}

% Credit to Gabriel Devenyi for this bibliography cfg:
% github.com/gdevenyi/mcmaster.latex
\usepackage[ style=numeric-comp, backend=biber, sorting=none, backref=true,
  maxnames=99, alldates=iso, seconds=true ]{biblatex} % bibliography
\addbibresource{../references.bib}

\usepackage{tabularx}
\usepackage{booktabs}

\title{Problem Statement and Goals\\\progname{}}

\author{\authname}

\date{}

%% Comments

\usepackage{color}

\newif\ifcomments\commentstrue %displays comments
%\newif\ifcomments\commentsfalse %so that comments do not display

% Changes: Use ``todonotes'' package for the comments, but don't use it for
% inlined template explanation information.

\ifcomments
    \newcommand{\plt}[1]{\textcolor{magenta}{[TPLT \textemdash{} #1]}}
    \usepackage[backgroundcolor=yellow,colorinlistoftodos]{todonotes}
    \setlength{\marginparwidth}{0.75in}
    \reversemarginpar % place on left-hand side
\else
    \newcommand{\plt}[1]{}
    \usepackage[disable]{todonotes}
\fi

\newcommand{\wss}[1]{\todo[color=blue]{SS: #1}}
\newcommand{\an}[1]{\todo[color=cyan]{\authinitials{}: #1}}

\newcounter{needscitationcounter}
\newcommand{\needscitation}{\stepcounter{needscitationcounter}\todo[color=red]{NC \theneedscitationcounter{}: Needs citation.}}
\newcommand{\nc}{\needscitation}

%% Common Parts

\newcommand{\progname}{BeamBending}
\newcommand{\authname}{\small{\textit{Team Drasil}}\\Jason Balaci}
\newcommand{\authinitials}{JB}

\usepackage{hyperref}
\hypersetup{colorlinks=true, linkcolor=blue, citecolor=blue, filecolor=blue,
            urlcolor=blue, unicode=false}
\urlstyle{same}


\begin{document}

\maketitle

\begin{table}[hp]
    \caption{Revision History} \label{TblRevisionHistory}
    \begin{tabularx}{\textwidth}{llX}
        \toprule
        \textbf{Date} & \textbf{Developer(s)} & \textbf{Change}                  \\
        \midrule
        Jan. 20, 2023 & Jason                 & Template created \& committed to
        git.
        \\
        Jan. 20, 2023 & Jason                 & Template filled in.              \\
        \bottomrule
    \end{tabularx}
\end{table}

\section{Problem Statement}
\label{problem-statement}

Beams safely support many constructions under load. Everything from bridges and
skyscrapers to commercial and residential properties use beams to safely carry
load by distributing the stress into their foundations and the ground\
\cite{Moscovitch2020}. Beams are flat, horizontal structural elements, that bear
load perpendicular to their horizon\ \cite{Moscovitch2020}.

Typically, inhabitants of residential constructions expect their floors to be
flat, balanced, and rigidly unmoving, or else they might feel uncomfortable in
their space. As such, beams must be rigidly fixed in place and be capable of
transferring all imposed loads down to the foundations of the buildings and the
ground. However, for other applications, such as bridges and beds of machine
tools, beams are free, within reason, to move horizontally\
\cite{BirdChivers1993}. \textit{Simply supported} beams are one kind of beams
that are commonly found in these other applications.

Simply supported beams use only two supports: a \textit{pinned} support and a
\textit{roller} support\ \cite{Lemonis2022}. The pinned support is fixed and
unmoving, while the roller support allows the beam to expand or contract axially
\cite{Lemonis2022}. To understand how simply supported beams handle uniformly
distributed loads, we aim to approximate the curve of deflection of a beam under
various conditions.

\subsection{Problem}
\label{problem-statement:problem}

Under simplified assumptions (i.e., Euler-Bernoulli beam theory
\cite{EulerBernoulliWiki}), to understand how various simply supported beams can
safely carry uniformly distributed loads, engineers must understand how beams
deflect and how the ends react.

\subsection{Inputs and Outputs}
\label{problem-statement:inputs-and-outputs}

As the scope of the work is relatively limited, the inputs and outputs are also
relatively limited. The inputs and outputs are as follows, following the
essential aspects of \cite{Lemonis2022}:

\subsubsection{Inputs}
\label{problem-statement:inputs-and-outputs:inputs}

\begin{enumerate}
    \item Beam structural properties:
          \begin{enumerate}
              \item length,
              \item material modulus of elasticity (Young's modulus), and
              \item moment of inertia.
          \end{enumerate}
    \item Magnitude of uniformly imposed force/load.
\end{enumerate}

\subsubsection{Outputs}

\begin{enumerate}
    \item At the supports, respectively:
          \begin{enumerate}
              \item force reactions, and
              \item angle of rotation.
          \end{enumerate}

    \item Related to the beam:
          \begin{enumerate}
              \item maximum deflection distance,
              \item bending moment, and
              \item transverse shear force.
          \end{enumerate}
\end{enumerate}

\subsection{Stakeholders}
\label{problem-statement:stakeholders}

As this work is conducted under CAS\ $741$, the stakeholders of this case study
includes Dr.\ Spencer Smith (who will be reviewing the work), myself (Jason
Balaci, who will be taking authorship of the work and whose grade is dependent
on this work), and peers in CAS\ $741$ (notably, reviewers). Additionally, as
this work is expected to be performed using, and contribute to, Drasil\
\cite{Drasil2023}, Dr.\ Jacques Carette is another stakeholder in the final
product and contributions to Drasil. Notably, as a colleague in CAS\ 741 and in
the Drasil Research Team, Sam Crawford is also a stakeholder in this work.
Finally, as this work will additionally become a case study under Drasil, the
greater community related to Drasil is also a stakeholder.

\subsection{Environment}
\label{problem-statement:environment}

As this work will be built using Drasil, it will be made in Haskell according to
the Haskell2010 language specification\ \cite{Haskell2010}, and it is limited to
producing the software artifacts that Drasil is able to generate. Since Drasil
already covers generating software artifacts for general-purpose programming
languages that compile and run on major operating systems and Central Processing
Unit (CPU) architectures, usability of the final software artifacts should not
be an issue to the majority of users. However, should foreign hardware or
software be desired as an environment, an extension should be built in Drasil to
allow for generating suitable software artifacts.

\section{Goals}
\label{goals}

This primary purpose of this work is to be a learning experience for myself,
Jason Balaci, about the development of Software Requirements Specifications
(SRS) documents and Scientific Computing Software (SCS). The secondary purpose
of this work is to create a new case study in Drasil whilst extending Drasil's
captured mathematical knowledge to include boundary value problems.

The tangible goals of this case study are as follows:

\begin{enumerate}
    \item to extend Drasil's mathematical knowledge (at least to include
          boundary value problems) and better understand it's current
          limitations, and
    \item to \textit{generate} a software artifact that conforms to the
          requirements as described by a well-formed SRS document abstraction.
          The SRS document abstraction will follow the needs of sufficiently
          describing the ``problem'' as described earlier in
          \autoref{problem-statement:problem}.
\end{enumerate}

\section{Stretch Goals}
\label{stretch-goals}

If time permits, there are also additional goals of this work, both related to
Drasil and the specific case study, with some overlap.


\begin{enumerate}
    \item Related to Drasil:
          \begin{enumerate}
              \item extend Drasil to generating GNU Octave or Julia code,
              \item create a guided, interactive tutorial on working with Drasil
                    (as discussed in \cite{BalaciDrasilIssueComment3194}),
              \item understanding the needs of building and manipulating
                    ``chunks'' \cite{Drasil2023} in Drasil, in the context of my
                    doctorate work (with particular focus on
                    \cite{BalaciDrasilDiscussion3003}),
              \item capture the idea of ``contexts'' to theories for Drasil to
                    be able to specialize/refine contextual quantities,
                    theories, and other chunks to particular contexts, and
              \item to become better acquainted with the chunk structure of
                    Drasil, potentially making contributions where applicable.
          \end{enumerate}

    \item Related to the case study:
          \begin{enumerate}
              \item build a family of case studies related to simply-supported
                    beams for \textit{various kinds of loads},
              \item build a family of case studies related to \textit{various
                        types of beams} for \textit{various kinds of loads},
              \item generate a visual diagram of the beam deflection at loading,
                    and finally,
              \item to solve the same problem with a system of linear equations
                    instead of the boundary-value problem formation.
          \end{enumerate}
\end{enumerate}

\newpage

\printbibliography[heading=bibintoc]

\end{document}
