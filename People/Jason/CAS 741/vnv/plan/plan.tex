\documentclass[12pt, titlepage]{article}

\usepackage{amsmath, mathtools}
\usepackage{amsfonts}
\usepackage{amssymb}
\usepackage{graphicx}
\graphicspath{{../assets}} % Bring in the assets


\usepackage{colortbl}
\usepackage{xr}
\usepackage{longtable}
\usepackage{xfrac}
\usepackage{tabularx}
\usepackage{float}
\usepackage{siunitx}
\usepackage{booktabs}
\usepackage{caption}
\usepackage{pdflscape}
\usepackage{afterpage}

\usepackage{hyperref}
\hypersetup{ colorlinks, citecolor=blue, filecolor=black, linkcolor=red,
    urlcolor=blue }
\usepackage{cleveref}

\usepackage{tikz}
\usepackage{varwidth}
\usetikzlibrary{shapes,arrows,cd,babel,arrows.meta,graphs,graphdrawing}
\usegdlibrary {layered}

% Credit to Gabriel Devenyi for this bibliography cfg:
% github.com/gdevenyi/mcmaster.latex
\usepackage[ style=numeric-comp, backend=biber, sorting=none, backref=true,
  maxnames=99, alldates=iso, seconds=true ]{biblatex} % bibliography
\addbibresource{../../references.bib}

%\usepackage{refcheck}


% Force todonotes to play nicely in other floats: credits to
% https://tex.stackexchange.com/a/402075
\usepackage{marginnote}
\renewcommand{\marginpar}{\marginnote}

%% Comments

\usepackage{color}

\newif\ifcomments\commentstrue %displays comments
%\newif\ifcomments\commentsfalse %so that comments do not display

% Changes: Use ``todonotes'' package for the comments, but don't use it for
% inlined template explanation information.

\ifcomments
    \newcommand{\plt}[1]{\textcolor{magenta}{[TPLT \textemdash{} #1]}}
    \usepackage[backgroundcolor=yellow,colorinlistoftodos]{todonotes}
    \setlength{\marginparwidth}{0.75in}
    \reversemarginpar % place on left-hand side
\else
    \newcommand{\plt}[1]{}
    \usepackage[disable]{todonotes}
\fi

\newcommand{\wss}[1]{\todo[color=blue]{SS: #1}}
\newcommand{\an}[1]{\todo[color=cyan]{\authinitials{}: #1}}

\newcounter{needscitationcounter}
\newcommand{\needscitation}{\stepcounter{needscitationcounter}\todo[color=red]{NC \theneedscitationcounter{}: Needs citation.}}
\newcommand{\nc}{\needscitation}

%% Common Parts

\newcommand{\progname}{BeamBending}
\newcommand{\authname}{\small{\textit{Team Drasil}}\\Jason Balaci}
\newcommand{\authinitials}{JB}

\usepackage{hyperref}
\hypersetup{colorlinks=true, linkcolor=blue, citecolor=blue, filecolor=blue,
            urlcolor=blue, unicode=false}
\urlstyle{same}

\usepackage{newunicodechar}

\newunicodechar{Ȧ}{$\dot{A}$}


\begin{document}

\title{System Verification and Validation Plan for \progname{}}
\author{\authname}
\date{\today}

\maketitle

\pagenumbering{roman}

\section{Revision History}

\begin{tabularx}{\textwidth}{p{3cm}p{2cm}X} \toprule {
    \bf Date} & {\bf Version} & {\bf Notes}      \\
    \midrule
    Feb 12    & 0.0           & Format template. \\
    \bottomrule
\end{tabularx}

\newpage

\tableofcontents

\listoftables
\plt{Remove this section if it isn't needed}

\listoffigures
\plt{Remove this section if it isn't needed}

\newpage

\section{Symbols, Abbreviations, and Acronyms}

\renewcommand{\arraystretch}{1.2}
\begin{tabular}{l l}
    \toprule
    \textbf{symbol} & \textbf{description} \\
    \midrule
    T               & Test                 \\
    \bottomrule
\end{tabular}\\

\plt{symbols, abbreviations, or acronyms --- you can simply reference the SRS
    tables, if appropriate}

\plt{Remove this section if it isn't needed}

\newpage

\pagenumbering{arabic}

This document ... \plt{provide an introductory blurb and roadmap of the
    Verification and Validation plan}

\section{General Information}

\subsection{Summary}

\plt{Say what software is being tested.  Give its name and a brief overview of
    its general functions.}

\subsection{Objectives}

\plt{State what is intended to be accomplished.  The objective will be around
    the qualities that are most important for your project.  You might have
    something like: ``build confidence in the software correctness,''
    ``demonstrate adequate usability.'' etc.  You won't list all of the
    qualities, just those that are most important.}

\subsection{Relevant Documentation}

\plt{Reference relevant documentation.  This will definitely include your SRS
    and your other project documents (design documents, like MG, MIS, etc).  You
    can include these even before they are written, since by the time the
    project is done, they will be written.}

\cite{BalaciBeamBendingSRS2023}

\section{Plan}

\plt{Introduce this section.   You can provide a roadmap of the sections to
    come.}

\subsection{Verification and Validation Team}

\plt{Your teammates.  Maybe your supervisor. You shoud do more than list names.
    You should say what each person's role is for the project's verification.  A
    table is a good way to summarize this information.}

\subsection{SRS Verification Plan}

\plt{List any approaches you intend to use for SRS verification.  This may
    include ad hoc feedback from reviewers, like your classmates, or you may
    plan for something more rigorous/systematic.}

\plt{Maybe create an SRS checklist?}

\subsection{Design Verification Plan}

\plt{Plans for design verification}

\plt{The review will include reviews by your classmates}

\plt{Create a checklists?}

\subsection{Verification and Validation Plan Verification Plan}

\plt{The verification and validation plan is an artifact that should also be verified.}

\plt{The review will include reviews by your classmates}

\plt{Create a checklists?}

\subsection{Implementation Verification Plan}

\plt{You should at least point to the tests listed in this document and the unit
    testing plan.}

\plt{In this section you would also give any details of any plans for static
    verification of the implementation.  Potential techniques include code
    walkthroughs, code inspection, static analyzers, etc.}

\subsection{Automated Testing and Verification Tools}

\plt{What tools are you using for automated testing.  Likely a unit testing
    framework and maybe a profiling tool, like ValGrind.  Other possible tools
    include a static analyzer, make, continuous integration tools, test coverage
    tools, etc.  Explain your plans for summarizing code coverage metrics.
    Linters are another important class of tools.  For the programming language
    you select, you should look at the available linters.  There may also be
    tools that verify that coding standards have been respected, like flake9 for
    Python.}

\plt{If you have already done this in the development plan, you can point to
    that document.}

\plt{The details of this section will likely evolve as you get closer to the
    implementation.}

\subsection{Software Validation Plan}

\plt{If there is any external data that can be used for validation, you should
    point to it here.  If there are no plans for validation, you should state
    that here.}

\plt{You might want to use review sessions with the stakeholder to check that
    the requirements document captures the right requirements.  Maybe task based
    inspection?}

\plt{This section might reference back to the SRS verification section.}

\section{System Test Description}

\subsection{Tests for Functional Requirements}

\plt{Subsets of the tests may be in related, so this section is divided into
    different areas.  If there are no identifiable subsets for the tests, this
    level of document structure can be removed.}

\plt{Include a blurb here to explain why the subsections below cover the
    requirements.  References to the SRS would be good here.}

\subsubsection{Area of Testing1}

\plt{It would be nice to have a blurb here to explain why the subsections below
    cover the requirements.  References to the SRS would be good here.  If a
    section covers tests for input constraints, you should reference the data
    constraints table in the SRS.}

\paragraph{Title for Test}

\begin{enumerate}

    \item{test-id1\\}

    Control: Manual versus Automatic

    Initial State:

    Input:

    Output: \plt{The expected result for the given inputs}

    Test Case Derivation: \plt{Justify the expected value given in the Output
        field}

    How test will be performed:

    \item{test-id2\\}

    Control: Manual versus Automatic

    Initial State:

    Input:

    Output: \plt{The expected result for the given inputs}

    Test Case Derivation: \plt{Justify the expected value given in the Output
        field}

    How test will be performed:

\end{enumerate}

\subsubsection{Area of Testing2}

...

\subsection{Tests for Nonfunctional Requirements}

\plt{The nonfunctional requirements for accuracy will likely just reference the
    appropriate functional tests from above.  The test cases should mention
    reporting the relative error for these tests.  Not all projects will
    necessarily have nonfunctional requirements related to accuracy}

\plt{Tests related to usability could include conducting a usability test and
    survey.  The survey will be in the Appendix.}

\plt{Static tests, review, inspections, and walkthroughs, will not follow the
    format for the tests given below.}

\subsubsection{Area of Testing1}

\paragraph{Title for Test}

\begin{enumerate}

    \item{test-id1\\}

    Type: Functional, Dynamic, Manual, Static etc.

    Initial State:

    Input/Condition:

    Output/Result:

    How test will be performed:

    \item{test-id2\\}

    Type: Functional, Dynamic, Manual, Static etc.

    Initial State:

    Input:

    Output:

    How test will be performed:

\end{enumerate}

\subsubsection{Area of Testing2}

...

\subsection{Traceability Between Test Cases and Requirements}

\plt{Provide a table that shows which test cases are supporting which
    requirements.}

\section{Unit Test Description}

\plt{Reference your MIS (detailed design document) and explain your overall
    philosophy for test case selection.} \plt{This section should not be filled
    in until after the MIS (detailed design document) has been completed.}

\subsection{Unit Testing Scope}

\plt{What modules are outside of the scope.  If there are modules that are
    developed by someone else, then you would say here if you aren't planning on
    verifying them.  There may also be modules that are part of your software,
    but have a lower priority for verification than others.  If this is the
    case, explain your rationale for the ranking of module importance.}

\subsection{Tests for Functional Requirements}

\plt{Most of the verification will be through automated unit testing.  If
    appropriate specific modules can be verified by a non-testing based
    technique.  That can also be documented in this section.}

\subsubsection{Module 1}

\plt{Include a blurb here to explain why the subsections below cover the module.
    References to the MIS would be good.  You will want tests from a black box
    perspective and from a white box perspective.  Explain to the reader how the
    tests were selected.}

\begin{enumerate}

    \item{test-id1\\}

    Type: \plt{Functional, Dynamic, Manual, Automatic, Static etc. Most will be
        automatic}

    Initial State:

    Input:

    Output: \plt{The expected result for the given inputs}

    Test Case Derivation: \plt{Justify the expected value given in the Output
        field}

    How test will be performed:

    \item{test-id2\\}

    Type: \plt{Functional, Dynamic, Manual, Automatic, Static etc. Most will be
        automatic}

    Initial State:

    Input:

    Output: \plt{The expected result for the given inputs}

    Test Case Derivation: \plt{Justify the expected value given in the Output
        field}

    How test will be performed:

    \item{...\\}

\end{enumerate}

\subsubsection{Module 2}

...

\subsection{Tests for Nonfunctional Requirements}

\plt{If there is a module that needs to be independently assessed for
    performance, those test cases can go here.  In some projects, planning for
    nonfunctional tests of units will not be that relevant.}

\plt{These tests may involve collecting performance data from previously
    mentioned functional tests.}

\subsubsection{Module ?}

\begin{enumerate}

    \item{test-id1\\}

    Type: \plt{Functional, Dynamic, Manual, Automatic, Static etc. Most will be
        automatic}

    Initial State:

    Input/Condition:

    Output/Result:

    How test will be performed:

    \item{test-id2\\}

    Type: Functional, Dynamic, Manual, Static etc.

    Initial State:

    Input:

    Output:

    How test will be performed:

\end{enumerate}

\subsubsection{Module ?}

...

\subsection{Traceability Between Test Cases and Modules}

\plt{Provide evidence that all of the modules have been considered.}

%%%%%%%%%%%%%%%%%%%%%%%%%%%%%%%%%%%%%%%%%%%%%%%%%%%%%%%%%%%%%%%%%%%%%%%%%%%%%%%
% Bibliography
%%%%%%%%%%%%%%%%%%%%%%%%%%%%%%%%%%%%%%%%%%%%%%%%%%%%%%%%%%%%%%%%%%%%%%%%%%%%%%%

\printbibliography

\newpage

\section{Appendix}

This is where you can place additional information.

\subsection{Symbolic Parameters}

The definition of the test cases will call for SYMBOLIC\_CONSTANTS. Their values
are defined in this section for easy maintenance.

\subsection{Usability Survey Questions?}

\plt{This is a section that would be appropriate for some projects.}

\newpage{}
\section*{Appendix --- Reflection}

The information in this section will be used to evaluate the team members on the
graduate attribute of Lifelong Learning.  Please answer the following questions:

\newpage{}
\section*{Appendix --- Reflection}

The information in this section will be used to evaluate the team members on the
graduate attribute of Lifelong Learning.  Please answer the following questions:

\begin{enumerate}
    \item What knowledge and skills will the team collectively need to acquire
          to successfully complete the verification and validation of your
          project? Examples of possible knowledge and skills include dynamic
          testing knowledge, static testing knowledge, specific tool usage etc.
          You should look to identify at least one item for each team member.
    \item For each of the knowledge areas and skills identified in the previous
          question, what are at least two approaches to acquiring the knowledge
          or mastering the skill?  Of the identified approaches, which will each
          team member pursue, and why did they make this choice?
\end{enumerate}

\end{document}
