\documentclass[12pt, titlepage]{article}

\usepackage{amsmath, mathtools}
\usepackage{amsfonts}
\usepackage{amssymb}
\usepackage{graphicx}
\graphicspath{{../assets}} % Bring in the assets


\usepackage{colortbl}
\usepackage{xr}
\usepackage{longtable}
\usepackage{xfrac}
\usepackage{tabularx}
\usepackage{float}
\usepackage{siunitx}
\usepackage{booktabs}
\usepackage{multirow}
\usepackage{caption}
\usepackage{pdflscape}
\usepackage{afterpage}
\usepackage[dvipsnames,table,xcdraw]{xcolor}

\usepackage{hyperref}
\hypersetup{ colorlinks, citecolor=blue, filecolor=black, linkcolor=red,
    urlcolor=blue }
\usepackage{cleveref}

\usepackage{tikz}
\usepackage{varwidth}
\usetikzlibrary{shapes,arrows,cd,babel,arrows.meta,graphs,graphdrawing}
\usegdlibrary {layered}

% Credit to Gabriel Devenyi for this bibliography cfg:
% github.com/gdevenyi/mcmaster.latex
\usepackage[ style=numeric-comp, backend=biber, sorting=none, backref=true,
  maxnames=99, alldates=iso, seconds=true ]{biblatex} % bibliography
\addbibresource{../../references.bib}

%\usepackage{refcheck}


% Force todonotes to play nicely in other floats: credits to
% https://tex.stackexchange.com/a/402075
\usepackage{marginnote}
\renewcommand{\marginpar}{\marginnote}

%% Comments

\usepackage{color}

\newif\ifcomments\commentstrue %displays comments
%\newif\ifcomments\commentsfalse %so that comments do not display

\ifcomments
\newcommand{\authornote}[3]{\textcolor{#1}{[#3 ---#2]}}
\newcommand{\todo}[1]{\textcolor{red}{[TODO: #1]}}
\else
\newcommand{\authornote}[3]{}
\newcommand{\todo}[1]{}
\fi

\newcommand{\wss}[1]{\authornote{blue}{SS}{#1}} 
\newcommand{\plt}[1]{\authornote{magenta}{TPLT}{#1}} %For explanation of the template
\newcommand{\an}[1]{\authornote{cyan}{Author}{#1}}

%% Common Parts

\newcommand{\progname}{BeamBending}
\newcommand{\authname}{\small{\textit{Team Drasil}}\\Jason Balaci}
\newcommand{\authinitials}{JB}

\usepackage{hyperref}
\hypersetup{colorlinks=true, linkcolor=blue, citecolor=blue, filecolor=blue,
            urlcolor=blue, unicode=false}
\urlstyle{same}

\usepackage{newunicodechar}

\newunicodechar{Ȧ}{$\dot{A}$}


\begin{document}

%------------------------------------------------------------------------------
% TITLE
%------------------------------------------------------------------------------

\title{System Verification and Validation Plan for \progname{}}
\author{\authname}
\date{\today}

\maketitle

\pagenumbering{roman}

%------------------------------------------------------------------------------
% REVISION HISTORY
%------------------------------------------------------------------------------

\section{Revision History}

\begin{tabularx}{\textwidth}{p{3cm}p{2cm}X} \toprule {
    \bf Date} & {\bf Version} & {\bf Notes}                                               \\
    \midrule
    Feb 12    & 0.0           & Format template.                                          \\
    Feb 13    & 0.1           & Preliminary work, read through and filling in easy spots. \\
    Feb 14    & 0.2           & Preliminary copy of ``general information'' section.      \\
    Feb 14    & 0.3           & Preliminary copy of ``plan'' section.                     \\
    Feb 14    & 0.4           & Preliminary dubious ``system'' (unit) tests.              \\
    Feb 16    & 0.5           & I guess the system tests weren't as dubious as I thought! Cleaning up as per in-class feedback ($E$, $I$, zeroes) \\
    Feb 17    & 0.6           & Complete draft.                                           \\
    \bottomrule
\end{tabularx}

%------------------------------------------------------------------------------
% TABLE OF CONTENTS
%------------------------------------------------------------------------------
\newpage

\tableofcontents

%------------------------------------------------------------------------------
% TABLE OF TABLES
%------------------------------------------------------------------------------
\newpage{}

\listoftables

%------------------------------------------------------------------------------
% TABLE OF SYMBOLS, ABBREVIATIONS, AND ACRONYMS
%------------------------------------------------------------------------------

\newpage

\section{Symbols, Abbreviations, and Acronyms}

The Symbols, Abbreviations, and Acronyms in this document builds upon those from
\progname{}'s related SRS document \cite{BalaciBeamBendingSRS2023}.\\

\renewcommand{\arraystretch}{1.2}
\begin{tabular}{l l}
    \toprule
    \textbf{Symbol} & \textbf{Description}                                    \\
    \midrule
    CAS             & Computing and Software department (McMaster University) \\
    SRS             & Software Requirements Specification                     \\
    T               & Test                                                    \\
    VnV             & Verification and Validation                             \\
    \bottomrule
\end{tabular}\\

\newpage

\pagenumbering{arabic}

%------------------------------------------------------------------------------
% GENERAL INFORMATION
%------------------------------------------------------------------------------

\section{General Information}

This document describes the plan of action related to the Verification and
Validation (VnV) of the Beam Bending analysis program (\progname{}). This VnV
plan will describe a plan of action for \textit{validating} that the Software
Requirements Specification (SRS)~\cite{BalaciBeamBendingSRS2023} for \progname{}
satisfies stakeholders, and \textit{verifying} that a supposedly conforming
software does indeed accurately satisfy the requirements.

\subsection{Summary}

The BeamBending Software Requirements Specification
(SRS)~\cite{BalaciBeamBendingSRS2023} describes the requirements of a
hypothetical program that analyzes beam deflection under imposed, distributed
loads on a simply-supported beam.

\subsection{Objectives}

The objective of this document is to outline a plan of action for:

\begin{enumerate}

    \item \textit{auditing} a continuously developed SRS document
          \cite{ParnasAndClements1986} for logical consistency,

    \item \textit{validating} said SRS satisfies stakeholder requirements, and

    \item \textit{verifying} that a produced software artifact conforms
          everything laid out in the SRS document (including, but not limited
          to, the functional and nonfunctional requirements), through both
          transparent and opaque testing.

\end{enumerate}

In doing this, we hope to build confidence in the coherence of software
requirements, and correctness and conformance of a software to said
specifications.

\subsection{Relevant Documentation}

As we intend to build (generate) the software with Drasil, the only relevant
documentation is that which is originally manually built, including:

\begin{enumerate}

    \item the SRS document \cite{BalaciBeamBendingSRS2023}, and

    \item this VnV plan.

\end{enumerate}

\noindent{}When the \progname{} program is re-created in Drasil, we may think of
that as a sort of ``documentation'' that we can to the above list.

%------------------------------------------------------------------------------
% PLAN
%------------------------------------------------------------------------------
\newpage{}

\section{Plan}

The ``whole'' Verification and Validation plan for \progname{} consists of
multiple sub-plans. Notably, it has a designated team (with sub-teams) who will
be executing the related sub-plans stipulated in this document. Team members
will take responsibility for various aspects of verification and validation.

\subsection{Verification and Validation Team}

Roles (\Cref{table:vnv_roles}) are assigned to each team member
(\Cref{table:vnv_teammates}), dictating the minimum responsibilities each member
has for each related project.

\begin{longtable}{|r|p{8cm}|}
    \caption{Table of VnV Roles}
    \label{table:vnv_roles}

    \\ \hline
    \rowcolor{Maroon}
    \textbf{Role}        & \textbf{Description/responsibilities}                                           \\ \hline
    \rowcolor{White}
    Supervisor           & Manager of all review committees, and distinguished reviewer and domain expert. \\ \hline
    Domain Expert        & Reviewer with considerable knowledge on underlying domains.                     \\ \hline
    Author               & Writer.                                                                         \\ \hline
    Reviewer             & Ensures documents are logically coherent and well-formed.                       \\ \hline
    Verifier             & Assures Drasil encoding of SRS accurately re-creates the manually created SRS.  \\ \hline
    Validator            & Assures SRS satisfies stakeholder requirements.                                 \\ \hline
    VnV-er               & Verifier \(\cup\) Validator.                                                    \\ \hline
\end{longtable}

\begin{longtable}{|r|c|l|}
    \caption{Table of VnV Teammates}
    \label{table:vnv_teammates}

    \\ \hline
    \rowcolor{Maroon}
    \textbf{Assignee}    & \textbf{Project}                  & \textbf{Role(s)}                                              \\ \hline
    \rowcolor{White}
    Dr.\ Smith           & \(*\)\footnote{\(*\): match all.} & Supervisor.                                                   \\ \hline
    Jason Balaci         & \(*\)                             & Author.                                                       \\ \hline
    Sam Crawford         & \(*\)                             & Domain Expert, Reviewer, and VnV-er.                          \\ \hline
    Mina Mahdipour       & SRS                               & Reviewer.                                                     \\ \hline
    Deesha Patel         & VnV                               & Reviewer.                                                     \\ \hline
    Maryam Valian        & Drasil                            & Reviewer \& VnV-er.                                           \\ \hline
    Class of CAS \(741\) & \(*\)                             & \(*\)                                                         \\ \hline
\end{longtable}

\subsection{SRS Verification Plan}
\label{srs-verification-plan}

In addition to checking that \progname{}'s Software Requirements Specification
(SRS) conforms to Dr.\ Smith's provided SRS checklist, we will have:

\begin{enumerate}

    \item a designated reviewing committee with a supervisor,

    \item a public presentation with a reviewing audience,

    \item built the project in Drasil, where we can build automated consistency
          checks and generate certain aspects of the document to avoid error,

    \item at least one external reviewer (Dr.\ Jacques Carette of the Drasil
          project) when the whole \progname{} project is sent to the main Drasil
          repository for merging, and finally,

    \item regular updates and sporadic reviews by current and future Drasil team
          members and onlookers (assuming the project is merged as an official
          case study of Drasil).

\end{enumerate}

A further external audit may be needed if the software is to be used in
non-educational applications\footnote{Note that this software, documentation,
and the likes is purely for educational purposes and hence comes with no
warranty and no liability by the authors.}. As this is out of scope, we will
provide no instruction.

\subsection{Design Verification Plan}

The software design does not need verification as the design of the software
will be based on Drasil's existing software family generator~\cite{Drasil2023}.
However, in order to build the Boundary Value Problem (BVP) in Drasil, we will
need to extend Drasil to generate BVP solving methods\footnote{This will be done
    and assumed as ``trusted'' when accepted into Drasil's main
    codebase~\cite{Drasil2023}.}. The onus of Drasil's validation is up to the
Drasil team\footnote{Including, but not limited to, Dr.\ Spencer Smith, Sam
    Crawford, and Jason Balaci.}.

\subsection{Verification and Validation Plan Verification Plan}

To assure that the Verification and Validation Plan adequately tests both the
SRS document and the relevant software, we will largely assume the ``many eyes''
hypothesis~\cite{Caraco1980avian} with many ``eyes'' of different skill-sets and
academic backgrounds (see \Cref{table:vnv_teammates}). Each team member should
test that this document conforms to the general VnV Checklist
document~\cite{SmithCapTemplate}.

\subsection{Implementation Verification Plan}

A proof of concept should be built and manually tested. When the project is
recreated in Drasil, the generated software artifacts should be similar to the
proof of concept, up to code style and organization. The generated software
artifacts should be tested against the manually created artifacts for
non-trivial or significant differences (to assure there are none\footnote{Unless
the manually created artifacts had problems, of course.}). Additionally, as
Drasil does not yet generate unit tests\footnote{But Sam might fix this for
us!}, the unit tests will be ported over to the generated software artifacts.

By re-writing the SRS with Drasil, the software implementation will be
generated. We have faith in the Drasil work, and so, the ``Implementation
Verification Plan'' is largely a ``Solution Validation Plan'' with an extra set
of requirements for the inputs and configuration of Drasil's code generator to
also be validated. 

\textit{Disclaimer: the scheme for auditing the Drasil-encoding of the SRS, and
the Drasil-generated solutions is fairly conventional (or so I believe), and so
is reiterated here for educational purposes.}

After the SRS has been verified (\Cref{srs-verification-plan}), the ``Drasil''
aspect of the project may be similarly verified by peer review, code
walkthrough, external audit, audit by assigned reviewers (see
\Cref{table:vnv_roles}), and audit by the supervisor (Dr.~Smith). However, the
focus of this will need to shift towards assuring that the encoding of the SRS
is of non-trivial depth and breadth\footnote{Whereby non-trivial structure of
the knowledge is captured by the encoding. See Chapter 2 of \cite{Balaci2022MSc}
for a ``deeper'' explanation.} and accurately represents the original SRS. The
code walkthrough in particular should be done with fellow Drasil researchers to
further assure that the capture of knowledge is indeed accurate and of
sufficient depth and breadth. 

By auditing the encoding of the SRS, we are effectively auditing the ``input''
given to Drasil. The configuration requirements for Drasil's code generator are
as follows:

\begin{enumerate}

    \item generate code:

          \begin{enumerate}

              \item in Python,

              \item with full code comment coverage\footnote{The ratio of the
                        number of well-documented ``code'' components to the
                        number of ``code'' components, where a code component is
                        defined as any logical component of a codebase (such as
                        functions, data types, classes, etc.).}, and

              \item ``full'' modularity\footnote{The generated software artifact
                        should be broken up into multiple logically grouped
                        software artifacts.},

          \end{enumerate}

    \item generate a Makefile with all common usage types (e.g., build, run,
          deps) as targets,

    \item generate basic usage documentation, and

    \item generate SRS artifacts from the same pool of knowledge used to build
          the previous two components.

\end{enumerate}

The configuration requirements should be verified similarly by all reviewers.

\subsection{Automated Testing and Verification Tools}

The reference code implementation and final generated code artifacts will be
tested (along with code coverage) using \textbf{pytest}~\cite{PyTest} to
automatically test the code against a series of unit tests (see
\Cref{sec:unit_test_description}). Continuous integration will be used to assure
that changes to the SRS encoding in Drasil does not change against the
well-tested artifacts\footnote{These are captured in the ``stable'' folder in
    Drasil's code repository, where ``stable'' artifacts remain manually tested.}.
The Python code will be aggressively formatted with Black~\cite{PythonBlack}.

\subsection{Software Validation Plan}

As the problem described in the SRS is similar to beam deflection problems
commonly found in engineering textbooks (such as \cite{BeerJohnston1981}), we
will assume a potential stakeholder is a writer of one of said textbooks.
Dr.~Spencer Smith will also be an assumed stakeholder in the project as he
suggested this project to the author. ``Input,'' ``output,'' and
``theory''-focused inspection will primarily be done to ensure that that
information contained in the SRS and the software satisfies stakeholders.

%------------------------------------------------------------------------------
% SYSTEM TEST DESCRIPTION 
%------------------------------------------------------------------------------
\newpage{}

\section{System Test Description}

\subsection{Tests for Functional Requirements}

The tests for functional requirements may be split up into 3 categories, as
follows:

\begin{enumerate}
    
    \item testing that inputs match the understood inputs
          (R2~\cite{BalaciBeamBendingSRS2023}),

    \item testing that the BVP solver functions as expected, and
    
    \item testing that the whole program accurately follows the instance models
          as described in the SRS (R3~\cite{BalaciBeamBendingSRS2023}).
    
\end{enumerate}

\subsubsection{Testing Inputs Are Outputted Accurately}

Throughout the next two categories of testing the functional requirements, we
will have tests on the program, and each test should be additionally
automatically checked that the re-iterated outputs match the intended inputs.

\subsubsection{Testing BVP Solver}

All of the tests for testing the BVP solver (\Cref{tab:simple_automatic_tests})
are done \textit{automatically} with a trivially ``empty'' initial state (e.g.,
the program is not started and has been provided no inputs yet), and trivial
inputs other than the BVPs themselves. The focus of this section is to test the
BVP solver. The inputs should be provided as appropriate and the expected output
should be printed. Each test will observe \(\forall x : \mathbb{R} ~.~x \in (0,
L_B) \Rightarrow (y(x) \approx_{\epsilon} y_a(x))\) (with pytest using samples
or symbolic equivalence, depending on solution). The expected outputs are
confirmed using WolframAlpha~\cite{WolframAlpha}.

Once all tests with trivial inputs are completed, all of the tests from
\Cref{tab:simple_automatic_tests} should be performed again\footnote{Note:
referencing here will be done with the BVP subscript removed, $T*$.} with
non-trivial $E_B$s and $I_B$s (e.g., not $1$). Since numeric scaling isn't very
consequential to the output, we will omit for brevity. Testing with randomized
inputs is a good strategy, similar to how testing is done via
\href{https://hackage.haskell.org/package/QuickCheck}{QuickCheck}, and should be
used with a reasonable range of values (see the Table of Software and Physical
constraints in the related SRS document). WolframAlpha may be similarly used as
a control, but having a trusted solver locally may be beneficial.

\begin{landscape}
    \begin{longtable}[c]{|l|ccc|c|l|}
        \caption{Simple, Automatic, Tests}
        \label{tab:simple_automatic_tests}                                                                                                                                                                                                                                                                                                                                          \\
        \hline
        \rowcolor{lightgray}
        \cellcolor{lightgray}                              & \multicolumn{3}{c|}{\cellcolor{lightgray}\textbf{Inputs}} & \textbf{Outputs}                              & \cellcolor{lightgray}                                                                                                                                                                                      \\ \cline{2-5}
        \rowcolor{lightgray}
        \multirow{-2}{*}{\cellcolor{lightgray}\textbf{ID}} & \multicolumn{1}{c|}{\cellcolor{lightgray}$w_B(x)$}          & \multicolumn{1}{c|}{\cellcolor{lightgray}$E_B$} & $I_B$                   & $y_a(x)$                                                                                                                 & \multirow{-2}{*}{\cellcolor{lightgray}\textbf{Control}} \\ \hline
        \endfirsthead
        \endhead
        T1$_{\text{BVP}}$                                                 & \multicolumn{1}{c|}{$0$}                                  & \multicolumn{1}{l|}{$1$}                      & $1$                   & $0$                                                                                                                      
                                                           & \href{https://www.wolframalpha.com/input?i=%5B%2F%2Fmath%3Asolve+y%27%27%27%27%3D0%2Cy%280%29%3D0%2Cy%2810%29%3D0%2Cy%27%27%280%29%3D0%2Cy%27%27%2810%29%3D0%2F%2F%5D}{WolframAlpha}                                                        \\ \hline
    
        T2$_{\text{BVP}}$                                                 & \multicolumn{1}{c|}{$1$}                                  & \multicolumn{1}{l|}{$1$}                      & $1$                   & \(\frac{x}{24} (x^3 - 20x^2 + 1 \text{E}^3)\)
                                                           & \href{https://www.wolframalpha.com/input?i=%5B%2F%2Fmath%3Asolve+y%27%27%27%27%3D1%2Cy%280%29%3D0%2Cy%2810%29%3D0%2Cy%27%27%280%29%3D0%2Cy%27%27%2810%29%3D0%2F%2F%5D}{WolframAlpha}                                                                                                                                                                                                                                                                                                                       \\ \hline
    
        T3$_{\text{BVP}}$                                                 & \multicolumn{1}{c|}{$-1$}                                 & \multicolumn{1}{l|}{$1$}                      & $1$                   & \(- \frac{x}{24} (x^3 - 20x^2 + 1 \text{E}^3)\)
                                                           & \href{https://www.wolframalpha.com/input?i=%5B%2F%2Fmath%3Asolve+y%27%27%27%27%3D-1%2Cy%280%29%3D0%2Cy%2810%29%3D0%2Cy%27%27%280%29%3D0%2Cy%27%27%2810%29%3D0%2F%2F%5D}{WolframAlpha}                                                                                                                                                                                                                                                                                                                       \\ \hline
    
        T4$_{\text{BVP}}$                                                 & \multicolumn{1}{c|}{$x$}                                  & \multicolumn{1}{l|}{$1$}                      & $1$                   & \(\frac{x}{360} (3x^4 - 1 \text{E}^3x^3 + 7 \text{E}^4)\)
                                                           & \href{https://www.wolframalpha.com/input?i=%5B%2F%2Fmath%3Asolve+y%27%27%27%27%3Dx%2Cy%280%29%3D0%2Cy%2810%29%3D0%2Cy%27%27%280%29%3D0%2Cy%27%27%2810%29%3D0%2F%2F%5D}{WolframAlpha}                                                                                                                                                                                                                                                                                                                       \\ \hline
    
        T5$_{\text{BVP}}$                                                 & \multicolumn{1}{c|}{$- x$}                                & \multicolumn{1}{l|}{$1$}                      & $1$                   & \(- \frac{x}{360} (3x^4 - 1 \text{E}^3x^3 + 7 \text{E}^4)\)
                                                           & \href{https://www.wolframalpha.com/input?i=%5B%2F%2Fmath%3Asolve+y%27%27%27%27%3D-x%2Cy%280%29%3D0%2Cy%2810%29%3D0%2Cy%27%27%280%29%3D0%2Cy%27%27%2810%29%3D0%2F%2F%5D}{WolframAlpha}                                                                                                                                                                                                                                                                                                                       \\ \hline
    
        T6$_{\text{BVP}}$                                                 & \multicolumn{1}{c|}{$8 \text{E}^4 x^3$}             & \multicolumn{1}{l|}{$1$}                      & $1$                   & \(\frac{2 \text{E}^4x}{21} (x^6 - 7 \text{E}^4x^2 + 6 \text{E}^6)\)
                                                           & \href{https://www.wolframalpha.com/input?i=%5B%2F%2Fmath%3Asolve+y%27%27%27%27%3D80000x%5E3%2Cy%280%29%3D0%2Cy%2810%29%3D0%2Cy%27%27%280%29%3D0%2Cy%27%27%2810%29%3D0%2F%2F%5D}{WolframAlpha}                                                                                                                                                                                                                                                                                                                       \\ \hline
    
        T7$_{\text{BVP}}$                                                 & \multicolumn{1}{c|}{$8 \text{E}^4 x^2$}                           & \multicolumn{1}{l|}{$1$}                      & $1$                   & \(\frac{2 \text{E}^3x}{9} (x^5 - 5 \text{E}^3x^4 + 4 \text{E}^5)\)
                                                           & \href{https://www.wolframalpha.com/input?i=%5B%2F%2Fmath%3Asolve+y%27%27%27%27%3D80000x%5E2%2Cy%280%29%3D0%2Cy%2810%29%3D0%2Cy%27%27%280%29%3D0%2Cy%27%27%2810%29%3D0%2F%2F%5D}{WolframAlpha}                                                                                                                                                                                                                                                                                                                       \\ \hline
    
        T8$_{\text{BVP}}$                                                 & \multicolumn{1}{c|}{$8 \text{E}^5 \sin{}(\frac{x\pi}{L})$}       & \multicolumn{1}{l|}{$1$}                      & $1$                   & \(\frac{1}{3\pi^{4}} (4 \text{E}^4L^2x(\pi^2(x^2 - 100) - 6L^2)\sin{}(\frac{10\pi}{L})+60L^2\sin{}(\frac{x\pi}{L}))\)
                                                           & \href{https://www.wolframalpha.com/input?i=%5B%2F%2Fmath%3Asolve+y%27%27%27%27%3D800000*sin%28%28x%2FL%29*pi%29%2Cy%280%29%3D0%2Cy%2810%29%3D0%2Cy%27%27%280%29%3D0%2Cy%27%27%2810%29%3D0%2F%2F%5D}{WolframAlpha}                                                                                                                                                                                                                                                                                                                       \\ \hline
    
        T9$_{\text{BVP}}$                                                 & \multicolumn{1}{c|}{$8 \text{E}^5 \sin{}(\frac{2x\pi}{L})$}      & \multicolumn{1}{l|}{$1$}                      & $1$                   & \(\frac{1}{3\pi^{4}} (5 \text{E}^3L^2x(2\pi^2(x^2 - 100) \linebreak{}\newline{}- 3L^2)\sin{}(\frac{20\pi}{L})+30L^2\sin{}(\frac{2x\pi}{L}))\)
                                                           & \href{https://www.wolframalpha.com/input?i=%5B%2F%2Fmath%3Asolve+y%27%27%27%27%3D800000*sin%28%282x%2FL%29*pi%29%2Cy%280%29%3D0%2Cy%2810%29%3D0%2Cy%27%27%280%29%3D0%2Cy%27%27%2810%29%3D0%2F%2F%5D}{WolframAlpha}                                                                                                                                                                                                                                                                                                                       \\ \hline
    
        % T10                                                & \multicolumn{1}{c|}{}                                     & \multicolumn{1}{l|}{$1$}                      & $1$                   &          
        % &                                                         \\ \hline
    
        % T11                                                & \multicolumn{1}{c|}{}                                     & \multicolumn{1}{l|}{$1$}                      & $1$                   &          
        % &                                                         \\ \hline
    \end{longtable}
\end{landscape}

\subsection{Tests for Nonfunctional Requirements}

The nonfunctional requirements are relatively uncomplicated to audit, mostly
because of the usage of Drasil:

\begin{enumerate}

    \item[$T_\text{NFR1}$] \textbf{Accuracy} is satisfied primarily through the
                tests of the functional requirements having a low
                tolerance\footnote{As this software is purely educational,
                accepting a higher tolerance is fine too.},

    \item[$T_\text{NFR2}$] \textbf{Usability} is strongly tied to Drasil's
                ability to generate code that can output
                data\footnote{Unfortunately, list-like functionality remains
                limited, but will be improved. Also note that ``usability'' was
                defined in the SRS document. Specifically, we will not be
                testing for accessibility nor any other facet as this software
                is meant to be an intermediate program used for calculation, not
                visualization.},

    \item[$T_\text{NFR3}$] \textbf{Maintainability} is satisfied through being
                constructed in Drasil, where changes in information have
                rippling effects and re-generation allows us to update
                everything to accomodate changes,

    \item[$T_\text{NFR4}$] \textbf{Portability} is satisfied because we aim to
                generate Python code, but also because all of Drasil's supported
                output languages are supported on the 3 major personal operating
                systems.

\end{enumerate}

\subsection{Traceability Between Test Cases and Requirements}

The following table traces the test cases as shown in the earlier sections back
to the functional and nonfunctional requirements\footnote{Note that $*$ is used
to quantify over each individual test case as it is redundant to have identical
rows for the tests that are each intended to test the same concepts.}.

\begin{longtable}{|c|c|c|c|c|c|c|c|c|c|}
    \caption{Tracing Tests to Requirements}
    \label{tab:trace_tests_to_reqs} \\
    
    \hline

                    & R1 & R2 & R3 & R4 & NFR1 & NFR2 & NFR3 & NFR4 \\ \hline
    $T*_\text{BVP}$ & X  & X  &    &    &      &      &      &      \\ \hline
    $T*$            & X  & X  & X  & X  &      &      &      &      \\ \hline
    $T_\text{NFR1}$ &    &    &    & X  & X    &      &      &      \\ \hline
    $T_\text{NFR2}$ &    &    &    &    &      & X    &      &      \\ \hline
    $T_\text{NFR3}$ &    &    &    &    &      &      & X    &      \\ \hline
    $T_\text{NFR4}$ &    &    &    &    &      &      &      & X    \\ \hline

\end{longtable}

%------------------------------------------------------------------------------
% UNIT TEST DESCRIPTION
%------------------------------------------------------------------------------
\newpage{}

\section{Unit Test Description}
\label{sec:unit_test_description}

As no software design documents will be constructed for Team Drasil's projects,
we will bootstrap the Drasil-generated software artifacts for testing. This
section will be filled in once we have Drasil generating code.

%%%%%%%%%%%%%%%%%%%%%%%%%%%%%%%%%%%%%%%%%%%%%%%%%%%%%%%%%%%%%%%%%%%%%%%%%%%%%%%
% Bibliography
%%%%%%%%%%%%%%%%%%%%%%%%%%%%%%%%%%%%%%%%%%%%%%%%%%%%%%%%%%%%%%%%%%%%%%%%%%%%%%%

\newpage{}

\printbibliography{}

%------------------------------------------------------------------------------
% APPENDIX
%------------------------------------------------------------------------------

\newpage{}

\section{Appendix}

\subsection{Symbolic Parameters}

In addition to the symbolic parameters from the SRS
document~\cite{BalaciBeamBendingSRS2023}, we will add \(\epsilon\), where
\(\epsilon = 10^{-3}\) (m), for usage as a tolerance for equivalence.

\subsection{Usability Survey Questions}

As the project will rely on Drasil to build the software from the requirement
description, any and all ``usability'' and/or ``accessibility'' concerns should
be directed towards the Drasil team as \progname{} will only use their basic
(stable) public-facing tooling.

\end{document}
