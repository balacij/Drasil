\documentclass[12pt, titlepage]{article}

\usepackage{amsmath, mathtools}
\usepackage{amsfonts}
\usepackage{amssymb}
\usepackage{graphicx}
\graphicspath{{../assets}} % Bring in the assets


\usepackage{colortbl}
\usepackage{xr}
\usepackage{longtable}
\usepackage{xfrac}
\usepackage{tabularx}
\usepackage{float}
\usepackage{siunitx}
\usepackage{booktabs}
\usepackage{caption}
\usepackage{pdflscape}
\usepackage{afterpage}
\usepackage[dvipsnames]{xcolor}

\usepackage{hyperref}
\hypersetup{ colorlinks, citecolor=blue, filecolor=black, linkcolor=red,
    urlcolor=blue }
\usepackage{cleveref}

\usepackage{tikz}
\usepackage{varwidth}
\usetikzlibrary{shapes,arrows,cd,babel,arrows.meta,graphs,graphdrawing}
\usegdlibrary {layered}

% Credit to Gabriel Devenyi for this bibliography cfg:
% github.com/gdevenyi/mcmaster.latex
\usepackage[ style=numeric-comp, backend=biber, sorting=none, backref=true,
  maxnames=99, alldates=iso, seconds=true ]{biblatex} % bibliography
\addbibresource{../../references.bib}

%\usepackage{refcheck}


% Force todonotes to play nicely in other floats: credits to
% https://tex.stackexchange.com/a/402075
\usepackage{marginnote}
\renewcommand{\marginpar}{\marginnote}

%% Comments

\usepackage{color}

\newif\ifcomments\commentstrue %displays comments
%\newif\ifcomments\commentsfalse %so that comments do not display

\ifcomments
\newcommand{\authornote}[3]{\textcolor{#1}{[#3 ---#2]}}
\newcommand{\todo}[1]{\textcolor{red}{[TODO: #1]}}
\else
\newcommand{\authornote}[3]{}
\newcommand{\todo}[1]{}
\fi

\newcommand{\wss}[1]{\authornote{blue}{SS}{#1}} 
\newcommand{\plt}[1]{\authornote{magenta}{TPLT}{#1}} %For explanation of the template
\newcommand{\an}[1]{\authornote{cyan}{Author}{#1}}

%% Common Parts

\newcommand{\progname}{BeamBending}
\newcommand{\authname}{\small{\textit{Team Drasil}}\\Jason Balaci}
\newcommand{\authinitials}{JB}

\usepackage{hyperref}
\hypersetup{colorlinks=true, linkcolor=blue, citecolor=blue, filecolor=blue,
            urlcolor=blue, unicode=false}
\urlstyle{same}

\usepackage{newunicodechar}

\newunicodechar{Ȧ}{$\dot{A}$}


\begin{document}

%------------------------------------------------------------------------------
% TITLE
%------------------------------------------------------------------------------

\title{System Verification and Validation Plan for \progname{}}
\author{\authname}
\date{\today}

\maketitle

\pagenumbering{roman}

%------------------------------------------------------------------------------
% REVISION HISTORY
%------------------------------------------------------------------------------

\section{Revision History}

\begin{tabularx}{\textwidth}{p{3cm}p{2cm}X} \toprule {
    \bf Date} & {\bf Version} & {\bf Notes}                                               \\
    \midrule
    Feb 12    & 0.0           & Format template.                                          \\
    Feb 13    & 0.1           & Preliminary work, read through and filling in easy spots. \\
    Feb 14    & 0.2           & Preliminary copy of ``general information'' section.      \\
    Feb 14    & 0.3           & Preliminary copy of ``plan'' section.                     \\
    \bottomrule
\end{tabularx}

%------------------------------------------------------------------------------
% TABLE OF CONTENTS
%------------------------------------------------------------------------------
\newpage

\tableofcontents

%------------------------------------------------------------------------------
% TABLE OF TABLES
%------------------------------------------------------------------------------
\newpage{}

\listoftables

%------------------------------------------------------------------------------
% TABLE OF SYMBOLS, ABBREVIATIONS, AND ACRONYMS
%------------------------------------------------------------------------------

\newpage

\section{Symbols, Abbreviations, and Acronyms}

The Symbols, Abbreviations, and Acronyms in this document builds upon those from
\progname{}'s related SRS document \cite{BalaciBeamBendingSRS2023}.\\

\renewcommand{\arraystretch}{1.2}
\begin{tabular}{l l}
    \toprule
    \textbf{Symbol} & \textbf{Description}                                    \\
    \midrule
    CAS             & Computing and Software department (McMaster University) \\
    SRS             & Software Requirements Specification                     \\
    T               & Test                                                    \\
    VnV             & Verification and Validation                             \\
    \bottomrule
\end{tabular}\\

\newpage

\pagenumbering{arabic}

%------------------------------------------------------------------------------
% GENERAL INFORMATION
%------------------------------------------------------------------------------

\section{General Information}

This document describes the plan of action related to the Verification and
Validation (VnV) of the Beam Bending analysis program (\progname{}). This VnV
plan will describe a plan of action for \textit{validating} that the Software
Requirements Specification (SRS)~\cite{BalaciBeamBendingSRS2023} for \progname{}
satisfies stakeholders, and \textit{verifying} that a supposedly conforming
software does indeed accurately satisfy the requirements.

\subsection{Summary}

The BeamBending Software Requirements Specification
(SRS)~\cite{BalaciBeamBendingSRS2023} derives the software requirements of a
hypothetical program that solves deflection analysis problems for
simply-supported beams. The focus of the program is to predict how
simply-supported beams handle imposed, distributed loads.

\subsection{Objectives}

The objective of this document is to outline a plan of action for:

\begin{enumerate}

    \item \textit{auditing} a continuously developed SRS document
          \cite{ParnasAndClements1986} for logical consistency,

    \item \textit{validating} said SRS satisfies stakeholder requirements, and

    \item \textit{verifying} that a produced software artifact conforms
          everything laid out in the SRS document (including, but not limited
          to, the functional and nonfunctional requirements), through both
          transparent and opaque testing.

\end{enumerate}

In doing this, we hope to build confidence in the coherence of software
requirements, and correctness and conformance of a software to said
specifications.

\subsection{Relevant Documentation}

As we intend to build (generate) the software with Drasil, the only relevant
documentation is that which is originally manually built, including:

\begin{enumerate}

    \item the SRS document \cite{BalaciBeamBendingSRS2023}, and

    \item this VnV plan.

\end{enumerate}

\noindent{}When the \progname{} program is re-created in Drasil, we may think of
that as a sort of ``documentation'' that we can to the above list.

%------------------------------------------------------------------------------
% PLAN
%------------------------------------------------------------------------------
\newpage{}

\section{Plan}

The ``whole'' Verification and Validation plan for \progname{} consists of
multiple sub-plans. Notably, it has a designated team (with sub-teams) who will
be executing the related sub-plans stipulated in this document. Team members
will take responsibility for various aspects of verification and validation.

\subsection{Verification and Validation Team}

Roles (\Cref{table:vnv_roles}) are assigned to each team member
(\Cref{table:vnv_teammates}), dictating the minimum responsibilities each member
has for each related project.

\begin{longtable}{|r|p{8cm}|}
    \caption{Table of VnV Roles}
    \label{table:vnv_roles}

    \\ \hline
    \rowcolor{Maroon}
    \textbf{Role}        & \textbf{Description/responsibilities}                                                       \\ \hline
    \rowcolor{White}
    Authoritative Expert & Manager of all review committees, and distinguished reviewer and domain expert.             \\ \hline
    Domain Expert        & Reviewer with considerable knowledge on underlying domains.                                 \\ \hline
    Author               & Writer.                                                                                     \\ \hline
    Reviewer             & Ensures documents are logically coherent and well-formed.                                   \\ \hline
    Verifier             & Assures Drasil encoding of SRS accurately re-creates the manually created SRS.\an{Odd\dots} \\ \hline
    Validator            & Assures SRS satisfies stakeholder requirements.                                             \\ \hline
    VnV-er               & Verifier \(\cup\) Validator.                                                                \\ \hline
\end{longtable}

\begin{longtable}{|r|c|l|}
    \caption{Table of VnV Teammates}
    \label{table:vnv_teammates}

    \\ \hline
    \rowcolor{Maroon}
    \textbf{Assignee}    & \textbf{Project}                  & \textbf{Role(s)}                                              \\ \hline
    \rowcolor{White}
    Dr.\ Smith           & \(*\)\footnote{\(*\): match all.} & Authoritative Expert\footnote{Pretend we're not in CAS 741.}. \\ \hline
    Jason Balaci         & \(*\)                             & Author.                                                       \\ \hline
    Sam Crawford         & \(*\)                             & Domain Expert, Reviewer, and VnV-er.                          \\ \hline
    Mina Mahdipour       & SRS                               & Reviewer.                                                     \\ \hline
    Deesha Patel         & VnV                               & Reviewer.                                                     \\ \hline
    Maryam Valian        & Drasil                            & Reviewer \& VnV-er.                                           \\ \hline
    Class of CAS \(741\) & \(*\)                             & \(*\)                                                         \\ \hline
\end{longtable}

\subsection{SRS Verification Plan}

In addition to checking that \progname{}'s Software Requirements Specification
(SRS) conforms to Dr.\ Smith's provided SRS checklist, we will have:

\begin{enumerate}

    \item a designated reviewing committee with a governing authoritative
          expert,

    \item a public presentation with a reviewing audience,

    \item built the project in Drasil, where we can build automated consistency
          checks and generate certain aspects of the document to avoid error,

    \item at least one external reviewer (Dr.\ Jacques Carette of the Drasil
          project) when the whole \progname{} project is sent to the main Drasil
          repository for merging, and finally,

    \item regular updates and sporadic reviews by current and future Drasil team
          members and onlookers (assuming the project is merged as an official
          case study of Drasil).

\end{enumerate}

\subsection{Design Verification Plan}

The software design does not need verification as the design of the software
will be based on Drasil's existing software family
generator~\cite{Drasil2023}\an{Well, I am extending Drasil a bit.}. The onus of
Drasil's validation is up to the Drasil team\footnote{Including, but not limited
    to, Dr.\ Spencer Smith, Sam Crawford, and Jason Balaci.}.

\subsection{Verification and Validation Plan Verification Plan}

To assure that the Verification and Validation Plan adequately tests both the
SRS document and the relevant software, we will largely assume the ``many eyes''
hypothesis~\cite{Caraco1980avian} with many ``eyes'' of different skill-sets and
academic backgrounds (see \Cref{table:vnv_teammates}). Each team member should
test that this document conforms to the general VnV Checklist
document~\cite{SmithCapTemplate}.

\subsection{Implementation Verification Plan}

A proof of concept should be built and manually tested. When the project is
re-created in Drasil, the generated software artifacts should be similar to the
proof of concept, up to code style and organization. The generated software
artifacts should be tested against the manually created artifacts if non-trivial
or significant differences exist. Additionally, as Drasil does not yet generate
unit tests\footnote{But Sam might fix this for us!}, the unit tests will be
ported over to the generated software artifacts.

By re-writing the SRS with Drasil, the software implementation will be
generated. We have faith in the Drasil work, and so, the ``Implementation
Verification Plan'' is largely a ``Solution Validation Plan'' with an extra set
of requirements for the configuration of Drasil's code generator. The solution
proposed in the SRS is to be validated by peer review, code walkthrough,
external audit, audit by assigned reviewers (see \Cref{table:vnv_roles}), and
audit by the authoritative figure (Dr.\ Smith). The configuration requirements
for Drasil's code generator are as follows:

\begin{enumerate}

    \item generate code:

          \begin{enumerate}

              \item in Python,

              \item with full code comment coverage\footnote{The ratio of the
                        number of well-documented ``code'' components to the
                        number of ``code'' components, where a code component is
                        defined as any logical component of a codebase (such as
                        functions, data types, classes, etc.).}, and

              \item ``full'' modularity\footnote{The generated software artifact
                        should be broken up into multiple logically grouped
                        software artifacts.},

          \end{enumerate}

    \item generate a Makefile with all common usage types (e.g., build, run,
          deps) as targets,

    \item generate basic usage documentation, and

    \item generate SRS artifacts from the same pool of knowledge used to build
          the previous two components.

\end{enumerate}

\subsection{Automated Testing and Verification Tools}

The reference code implementation and final generated code artifacts will be
tested (along with code coverage) using \textbf{pytest}~\cite{PyTest} to
automatically test the code against a series of unit tests (see
\Cref{sec:unit_test_description}). Continuous integration will be used to assure
that changes to the SRS encoding in Drasil does not change against the
well-tested artifacts\footnote{These are captured in the ``stable'' folder in
    Drasil's code repository, where ``stable'' artifacts remain manually tested.}.
The Python code will be aggressively formatted with Black~\cite{PythonBlack}.

\subsection{Software Validation Plan}

\plt{If there is any external data that can be used for validation, you should
    point to it here.  If there are no plans for validation, you should state
    that here.}

\plt{You might want to use review sessions with the stakeholder to check that
    the requirements document captures the right requirements.  Maybe task based
    inspection?}

\plt{This section might reference back to the SRS verification section.}


%------------------------------------------------------------------------------
% SYSTEM TEST DESCRIPTION 
%------------------------------------------------------------------------------
\newpage{}

\section{System Test Description}

\subsection{Tests for Functional Requirements}

\plt{Subsets of the tests may be in related, so this section is divided into
    different areas.  If there are no identifiable subsets for the tests, this
    level of document structure can be removed.}

\plt{Include a blurb here to explain why the subsections below cover the
    requirements.  References to the SRS would be good here.}

\injb{``Trivial'': \[ [//math:solve y''''=0,y(0)=0,y(10)=0,y''(0)=0,y''(10)=0//] \] }
\injb{``Trivial'': \[ [//math:solve y''''=1,y(0)=0,y(10)=0,y''(0)=0,y''(10)=0//] \] }
\injb{``Trivial'': \[ [//math:solve y''''=-1,y(0)=0,y(10)=0,y''(0)=0,y''(10)=0//] \] }
\injb{``Trivial'': \[ [//math:solve y''''=x,y(0)=0,y(10)=0,y''(0)=0,y''(10)=0//] \] }
\injb{``Trivial'': \[ [//math:solve y''''=-x,y(0)=0,y(10)=0,y''(0)=0,y''(10)=0//] \] }

\injb{``Trivial'': \[ [//math:solve y''''=80000x^3,y(0)=0,y(10)=0,y''(0)=0,y''(10)=0//] \] }
\injb{``Trivial'': \[ [//math:solve y''''=0x^3,y(0)=0,y(10)=0,y''(0)=0,y''(10)=0//] \] }
\injb{``Trivial'': \[ [//math:solve y''''=80000x^2,y(0)=0,y(10)=0,y''(0)=0,y''(10)=0//] \] }
\injb{``Trivial'': \[ [//math:solve y''''=0x^2,y(0)=0,y(10)=0,y''(0)=0,y''(10)=0//] \] }
\injb{``Trivial'': \[ [//math:solve y''''=10000000000x^3,y(0)=0,y(10)=0,y''(0)=0,y''(10)=0//] \] }

\injb{``Trivial'': \[ [//math:solve y''''=800000*sin((x/L)*pi),y(0)=0,y(10)=0,y''(0)=0,y''(10)=0//] \] }
\injb{``Trivial'': \[ [//math:solve y''''=800000*sin((x/L)*2*pi),y(0)=0,y(10)=0,y''(0)=0,y''(10)=0//] \] }


\subsubsection{Area of Testing1}

\plt{It would be nice to have a blurb here to explain why the subsections below
    cover the requirements.  References to the SRS would be good here.  If a
    section covers tests for input constraints, you should reference the data
    constraints table in the SRS.}

\paragraph{Title for Test}

\begin{enumerate}

    \item{test-id1\\}

    Control: Manual versus Automatic

    Initial State:

    Input:

    Output: \plt{The expected result for the given inputs}

    Test Case Derivation: \plt{Justify the expected value given in the Output
        field}

    How test will be performed:

    \item{test-id2\\}

    Control: Manual versus Automatic

    Initial State:

    Input:

    Output: \plt{The expected result for the given inputs}

    Test Case Derivation: \plt{Justify the expected value given in the Output
        field}

    How test will be performed:

\end{enumerate}

\subsubsection{Area of Testing2}

...

\subsection{Tests for Nonfunctional Requirements}

\plt{The nonfunctional requirements for accuracy will likely just reference the
    appropriate functional tests from above.  The test cases should mention
    reporting the relative error for these tests.  Not all projects will
    necessarily have nonfunctional requirements related to accuracy}

\plt{Tests related to usability could include conducting a usability test and
    survey.  The survey will be in the Appendix.}

\plt{Static tests, review, inspections, and walkthroughs, will not follow the
    format for the tests given below.}

\subsubsection{Area of Testing1}

\paragraph{Title for Test}

\begin{enumerate}

    \item{test-id1\\}

    Type: Functional, Dynamic, Manual, Static etc.

    Initial State:

    Input/Condition:

    Output/Result:

    How test will be performed:

    \item{test-id2\\}

    Type: Functional, Dynamic, Manual, Static etc.

    Initial State:

    Input:

    Output:

    How test will be performed:

\end{enumerate}

\subsubsection{Area of Testing2}

...

\subsection{Traceability Between Test Cases and Requirements}

\plt{Provide a table that shows which test cases are supporting which
    requirements.}


%------------------------------------------------------------------------------
% UNIT TEST DESCRIPTION
%------------------------------------------------------------------------------
\newpage{}

\section{Unit Test Description}
\label{sec:unit_test_description}

\plt{Reference your MIS (detailed design document) and explain your overall
    philosophy for test case selection.} \plt{This section should not be filled
    in until after the MIS (detailed design document) has been completed.}

\subsection{Unit Testing Scope}

\plt{What modules are outside of the scope.  If there are modules that are
    developed by someone else, then you would say here if you aren't planning on
    verifying them.  There may also be modules that are part of your software,
    but have a lower priority for verification than others.  If this is the
    case, explain your rationale for the ranking of module importance.}

\subsection{Tests for Functional Requirements}

\plt{Most of the verification will be through automated unit testing.  If
    appropriate specific modules can be verified by a non-testing based
    technique.  That can also be documented in this section.}

\subsubsection{Module 1}

\plt{Include a blurb here to explain why the subsections below cover the module.
    References to the MIS would be good.  You will want tests from a black box
    perspective and from a white box perspective.  Explain to the reader how the
    tests were selected.}

\begin{enumerate}

    \item{test-id1\\}

    Type: \plt{Functional, Dynamic, Manual, Automatic, Static etc. Most will be
        automatic}

    Initial State:

    Input:

    Output: \plt{The expected result for the given inputs}

    Test Case Derivation: \plt{Justify the expected value given in the Output
        field}

    How test will be performed:

    \item{test-id2\\}

    Type: \plt{Functional, Dynamic, Manual, Automatic, Static etc. Most will be
        automatic}

    Initial State:

    Input:

    Output: \plt{The expected result for the given inputs}

    Test Case Derivation: \plt{Justify the expected value given in the Output
        field}

    How test will be performed:

    \item{...\\}

\end{enumerate}

\subsubsection{Module 2}

...

\subsection{Tests for Nonfunctional Requirements}

\plt{If there is a module that needs to be independently assessed for
    performance, those test cases can go here.  In some projects, planning for
    nonfunctional tests of units will not be that relevant.}

\plt{These tests may involve collecting performance data from previously
    mentioned functional tests.}

\subsubsection{Module ?}

\begin{enumerate}

    \item{test-id1\\}

    Type: \plt{Functional, Dynamic, Manual, Automatic, Static etc. Most will be
        automatic}

    Initial State:

    Input/Condition:

    Output/Result:

    How test will be performed:

    \item{test-id2\\}

    Type: Functional, Dynamic, Manual, Static etc.

    Initial State:

    Input:

    Output:

    How test will be performed:

\end{enumerate}

\subsubsection{Module ?}

...

\subsection{Traceability Between Test Cases and Modules}

\plt{Provide evidence that all of the modules have been considered.}

%%%%%%%%%%%%%%%%%%%%%%%%%%%%%%%%%%%%%%%%%%%%%%%%%%%%%%%%%%%%%%%%%%%%%%%%%%%%%%%
% Bibliography
%%%%%%%%%%%%%%%%%%%%%%%%%%%%%%%%%%%%%%%%%%%%%%%%%%%%%%%%%%%%%%%%%%%%%%%%%%%%%%%

\newpage{}

\printbibliography{}

%------------------------------------------------------------------------------
% APPENDIX
%------------------------------------------------------------------------------

\newpage{}

\section{Appendix}

\subsection{Symbolic Parameters}

There are no symbolic constants (nor parameters) needed for \progname{}.

\subsection{Usability Survey Questions}

As the project will rely on Drasil to build the software from the requirement
description, any and all ``usability'' and/or ``accessibility'' concerns should
be directed towards the Drasil team as \progname{} will only use their basic
(stable) public-facing tooling.

\end{document}
